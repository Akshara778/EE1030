\let\negmedspace\undefined
\let\negthickspace\undefined
\documentclass[journal,12pt,onecolumn,article]{IEEEtran}
\usepackage{cite}
\usepackage{color,soul}
\usepackage{amsmath,amssymb,amsfonts,amsthm}
\usepackage{algorithmic}
\usepackage{graphicx}
\usepackage{textcomp}
\usepackage{xcolor}
\usepackage{txfonts}
\usepackage{listings}
\usepackage{enumitem}
\usepackage{mathtools}
\usepackage{gensymb}
\usepackage{comment}
\usepackage[breaklinks=true]{hyperref}
\usepackage{tkz-euclide} 
\usepackage{listings}
\usepackage{multicol}
\usepackage{gvv}       
\usepackage[dvipsnames]{xcolor}
\def\inputGnumericTable{}                                
\usepackage[latin1]{inputenc}                            
\usepackage{color}                                       
\usepackage{array}                                       
\usepackage{longtable}                                   
\usepackage{calc}                                        
\usepackage{multirow}                                    
\usepackage{hhline}                                      
\usepackage{ifthen}                                      
\usepackage{lscape}
\newtheorem{theorem}{Theorem}[section]
\newtheorem{problem}{Problem}
\newtheorem{proposition}{Proposition}[section]
\newtheorem{lemma}{Lemma}[section]
\newtheorem{corollary}[theorem]{Corollary}
\newtheorem{example}{Example}[section]
\newtheorem{definition}[problem]{Definition}
\newcommand{\BEQA}{\begin{eqnarray}}
\newcommand{\EEQA}{\end{eqnarray}}
\newcommand{\define}{\stackrel{\triangle}{=}}
\theoremstyle{remark}
\newtheorem{rem}{Remark}
\begin{document}
\bibliographystyle{IEEEtran}
\vspace{3cm}
\title{ASSIGNMENT 3}
\author{EE24BTECH11003 - Akshara Sarma Chennubhatla}
\maketitle
\section{D: MCQs with One or More than One Correct}
\begin{enumerate}[start=3]
\item The probability that at least one of the events A and B occurs is 0.6. If A and B occur simultaneously with probability 0.2, then $P\brak{\overline{A}}+P\brak{\overline{B}}$ is
\hfill{\brak{1987-2 Marks}}
\begin{enumerate}
\begin{multicols}{4}
\item 0.4
\columnbreak
\item 0.8
\columnbreak
\item 1.2
\columnbreak
\item 1.4
\end{multicols}
\item none
\end{enumerate}
(Here $\overline{A}$ and $\overline{B}$ are the complements of A and B, respectively).
\item For two given events A and B, $P\brak{A \cap B}$
\hfill{\brak{1988-2 Marks}}
\begin{enumerate}
\item not less than $P\brak{A}+P\brak{B}-1$
\item not greater than $P\brak{A}+P\brak{B}$
\item equal to $P\brak{A}+P\brak{B}-P\brak{A \cup B}$
\item equal to $P\brak{A}+P\brak{B}+P\brak{A \cup B}$
\end{enumerate}
\item If E and F are independent events such that $0<P\brak{E}<1$ and $0<P\brak{F}<1$, then 
\hfill{\brak{1989-2 Marks}}
\begin{enumerate}
\item E and F are mutually exclusive
\item E and $F^\complement$ (the complement of the event F) \\are independent
\item $E^\complement$ and $F^\complement$ are independent
\item $P\brak{E|F}+P\brak{E^\complement|F^\complement}=1$.
\end{enumerate}
\item For any two events A and B in a sample \\space
\hfill{\brak{1991-2 Marks}}
\begin{enumerate}
\item $P\brak{A|B}\geq \frac{P\brak{A}+P\brak{B}-1}{P\brak{B}}, P\brak{B}\neq 0$ is always\\ true
\item $P\brak{A\cap \overline{B}}=P\brak{A}-P\brak{A\cap B}$ does not \\hold
\item $P\brak{A\cup B}=1-P\brak{\overline{A}}P\brak{\overline{B}}$, if A and B \\are independent
\item $P\brak{A\cup B}=1-P\brak{\overline{A}}P\brak{\overline{B}}$, if A and B \\are disjoint.
\end{enumerate}
\item E and F are two independent events. The probability that both E and F happen is $\frac{1}{12}$ and the probability that neither E nor F happens is $\frac{1}{2}$. Then,
\hfill{\brak{1993-2 Marks}}
\begin{enumerate}
\item $P\brak{E}=\frac{1}{3},P\brak{F}=\frac{1}{4}$
\item $P\brak{E}=\frac{1}{2},P\brak{F}=\frac{1}{6}$
\item $P\brak{E}=\frac{1}{6},P\brak{F}=\frac{1}{2}$
\item $P\brak{E}=\frac{1}{4},P\brak{F}=\frac{1}{3}$
\end{enumerate}
\item Let $0<P\brak{A}<1,0<P\brak{B}<1$ and $P\brak{A\cup B}=P\brak{A}+P\brak{B}-P\brak{A}P\brak{B}$ then
\hfill{\brak{1995S}}
\begin{enumerate}
\item $P\brak{A|B}=P\brak{B}-P\brak{A}$
\item $P\brak{A'-B'}=P\brak{A'}-P\brak{B'}$
\item $P\brak{A\cup B}'=P\brak{A'}P\brak{B'}$
\item $P\brak{A|B}=P\brak{A}$
\end{enumerate}
\item If from each of the three boxes containing 3 white and 1 black, 2 white and 2 black, 1 white and 3 black balls, one ball is drawn at random, then the probability that 2 white and 1 black ball will be drawn is
\hfill{\brak{1998-2 Marks}}
\begin{enumerate}
\begin{multicols}{4}
\item $\frac{13}{32}$
\columnbreak
\item $\frac{1}{4}$
\columnbreak
\item $\frac{1}{32}$
\columnbreak
\item $\frac{3}{16}$
\end{multicols} 
\end{enumerate}
\item If $\overline{E}$ and $\overline{F}$ are the complementary events of events E and F respectively and if $0<P\brak{F}<1$, then
\hfill{\brak{1998-2 Marks}}
\begin{enumerate}
\item $P\brak{E|F}+P\brak{\overline{E}|F}=1$
\item $P\brak{E|F}+P\brak{E|\overline{F}}=1$
\item $P\brak{\overline{E}|F}+P\brak{E|\overline{F}}=1$
\item $P\brak{E|\overline{F}}+P\brak{\overline{E}|\overline{F}}=1$
\end{enumerate}
\item There are four machines and it is known that exactly two of them are faulty. They are tested, one by one, in a random order till both the faulty machines are identified. Then the probability that only two tests are needed is
\hfill{\brak{1998-2 Marks}}
\begin{enumerate}
\begin{multicols}{4}
\item $\frac{1}{3}$
\columnbreak
\item $\frac{1}{6}$
\columnbreak
\item $\frac{1}{2}$
\columnbreak
\item $\frac{1}{4}$
\end{multicols}
\end{enumerate}
\item If E and F are events with $P\brak{E} \leq P\brak{F}$ and $P\brak{E\cap F}>0$, then
\hfill{\brak{1998-2 Marks}}
\begin{enumerate}
\item occurrence of E $\Rightarrow$ occurrence of F
\item occurrence of F $\Rightarrow$ occurrence of E
\item non-occurrence of E $\Rightarrow$ non-occurrence \\of F
\item none of the above implications holds
\end{enumerate}
\item A fair coin is tossed repeatedly. If the tail appears on first four tosses, then the probability of the head appearing on the fifth toss equals
\hfill{\brak{1998-2 Marks}}
\begin{enumerate}
\begin{multicols}{4}
\item $\frac{1}{2}$
\columnbreak
\item $\frac{1}{32}$
\columnbreak
\item $\frac{31}{32}$
\columnbreak
\item $\frac{1}{5}$
\end{multicols}
\end{enumerate}
\item Seven white balls and three black balls are randomly placed in a row. The probability that no two black balls are placed adjacently equals
\hfill{\brak{1998-2 Marks}}
\begin{enumerate}
\begin{multicols}{4}
\item $\frac{1}{2}$
\columnbreak
\item $\frac{7}{15}$
\columnbreak
\item $\frac{2}{15}$
\columnbreak
\item $\frac{1}{3}$
\end{multicols}
\end{enumerate}
\item The probabilities that a student passes in Mathematics, Physics and Chemistry are m, p and c, respectively. Of these subjects, the student has a 75\% chance of passing in at least one, a 50\% chance of passing in at least two, and a 40\% chance of passing in exactly two. Which of the following relations are true?
\hfill{\brak{1999-3 Marks}}
\begin{enumerate}
\begin{multicols}{2}
\item $p+m+c=\frac{19}{20}$
\columnbreak
\item $p+m+c=\frac{27}{20}$
\end{multicols}
\begin{multicols}{2}
\item $pmc=\frac{1}{10}$
\columnbreak
\item $pmc=\frac{1}{4}$
\end{multicols}
\end{enumerate}
\item Let E and F be two independent events. The probability that exactly one of them occurs is $\frac{11}{25}$ and the probability of none of them occurring is $\frac{2}{25}$. If $P\brak{T}$ denotes the probability of occurrence of the event T, then
\hfill{\brak{2011}}
\begin{enumerate}
\begin{multicols}{2}
\item $P\brak{E}=\frac{4}{5},P\brak{F}=\frac{3}{5}$
\columnbreak
\item $P\brak{E}=\frac{1}{5},P\brak{F}\\=\frac{2}{5}$
\end{multicols}
\begin{multicols}{2}
\item $P\brak{E}=\frac{2}{5},P\brak{F}=\frac{1}{5}$
\columnbreak
\item $P\brak{E}=\frac{3}{5},P\brak{F}\\=\frac{4}{5}$
\end{multicols}
\end{enumerate}
\item A ship is fitted with three engines $E_1,E_2$ and $E_3$. The engines function independently of each other with respective probabilities $\frac{1}{2},\frac{1}{4}$ and $\frac{1}{4}$. For the ship to be operational at least two of its engines must function. Let $X$ denote the event that the ship is operational and let $X_1,X_2$ and $X_3$ denote respectively the events that the engines $E_1,E_2$ and $E_3$ are functioning. Which of the following is(are) true?
\hfill{\brak{2012}}
\begin{enumerate}
\item $P\sbrak{X_1^\complement|X}=\frac{3}{16}$
\item $P$[Exactly two engines of the ship are \\functioning\;$|$\; X]$=\frac{7}{8}$
\item $P\sbrak{X|X_2}=\frac{5}{16}$
\item $P\sbrak{X|X_1}=\frac{7}{16}$
\end{enumerate}
\end{enumerate}
\end{document}
