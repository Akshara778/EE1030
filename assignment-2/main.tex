\let\negmedspace\undefined
\let\negthickspace\undefined
\documentclass[journal,12pt,onecolumn,article]{IEEEtran}
\usepackage{cite}
\usepackage{color,soul}
\usepackage{amsmath,amssymb,amsfonts,amsthm}
\usepackage{algorithmic}
\usepackage{graphicx}
\usepackage{textcomp}
\usepackage{xcolor}
\usepackage{txfonts}
\usepackage{listings}
\usepackage{enumitem}
\usepackage{mathtools}
\usepackage{gensymb}
\usepackage{comment}
\usepackage[breaklinks=true]{hyperref}
\usepackage{tkz-euclide} 
\usepackage{listings}
\usepackage{multicol}
\usepackage{gvv}       
\usepackage[dvipsnames]{xcolor}
\def\inputGnumericTable{}                                
\usepackage[latin1]{inputenc}                            
\usepackage{color}                                       
\usepackage{array}                                       
\usepackage{longtable}                                   
\usepackage{calc}                                        
\usepackage{multirow}                                    
\usepackage{hhline}                                      
\usepackage{ifthen}                                      
\usepackage{lscape}
\newtheorem{theorem}{Theorem}[section]
\newtheorem{problem}{Problem}
\newtheorem{proposition}{Proposition}[section]
\newtheorem{lemma}{Lemma}[section]
\newtheorem{corollary}[theorem]{Corollary}
\newtheorem{example}{Example}[section]
\newtheorem{definition}[problem]{Definition}
\newcommand{\BEQA}{\begin{eqnarray}}
\newcommand{\EEQA}{\end{eqnarray}}
\newcommand{\define}{\stackrel{\triangle}{=}}
\theoremstyle{remark}
\newtheorem{rem}{Remark}
\begin{document}
\bibliographystyle{IEEEtran}
\vspace{3cm}
\title{ASSIGNMENT 2}
\author{EE24BTECH11003 - Akshara Sarma Chennubhatla}
\maketitle
\section{D: MCQs with One or More than One Correct}
\begin{enumerate}[start = 6]
\item Let $y\brak{x}$ be a solution of the differential equation $\brak{1+e^x}y^\prime + ye^x = 1$. If $y\brak{0} = 2$, then which of the following statement is (are) true?
\hfill{\brak{JEE Adv.2015}}
\begin{enumerate}
\begin{multicols}{2}
\item $y\brak{-4} = 0$
\columnbreak
\item $y\brak{-2} = 0$
\end{multicols}
\item $y\brak{x}$ has a critical point in the interval \\ $\brak{-1,0}$
\item $y\brak{x}$ has no critical point in the interval \\ $\brak{-1,0}$
\end{enumerate}
\item Consider the family of all circles whose centers lie on the straight line $y = x$. If this family of circle is represented by the differential equation $Py^{\prime\prime} + Qy^\prime + 1 = 0$, where P, Q are functions of $x,y$ and $y^\prime$
\begin{align}
\brak{here\;y^\prime = \frac{dy}{dx}, y^{\prime\prime} = \frac{d^2y}{dx^2}},
\end{align} then which of the following statements is (are) true?
\hfill{\brak{JEE Adv.2015}}
\begin{enumerate}
\begin{multicols}{2}
\item $P = y + x$
\columnbreak
\item $P = y-x$
\end{multicols}
\begin{multicols}{2}
\item $P+Q = 1-x+y\\+y^\prime+\brak{y^\prime}^2$
\columnbreak
\item $P-Q=x+y\\-y^\prime-\brak{y^\prime}^2$ 
\end{multicols}
\end{enumerate}
\item Let $f:\brak{0,\infty} \rightarrow \Re$ be a differentiable function such that 
\begin{align}
f^\prime\brak{x} = 2-\frac{f\brak{x}}{x}
\end{align} for all $x \in \brak{0,\infty} and f\brak{1} \neq 1$. \\Then
\hfill{\brak{JEE Adv.2016}}
\begin{enumerate}
\begin{multicols}{2}
\item $\lim\limits_{x \to 0+} f^\prime\brak{\frac{1}{x}} = 1$
\columnbreak
\item $\lim\limits_{x \to 0+} x f^\prime\brak{\frac{1}{x}} = 2$
\end{multicols}
\begin{multicols}{2}
\item $\lim\limits_{x \to 0+} x^2 f^\prime\brak{x} = 0$
\columnbreak
\item $\abs{f\brak{x}} \leq 2$ for all \\$x \in \brak{0,2}$ 
\end{multicols} 
\end{enumerate}
\item A solution curve of the differential equation 
\begin{align}
\brak{x^2+xy+4x+2y+4}\frac{dy}{dx}-y^2=0,\;x>0,
\end{align} passes through the point $\brak{1,3}$. Then the solution curve
\hfill{\brak{JEE Adv.2016}}
\begin{enumerate}
\item intersects $y=x+2$ exactly at one point
\item intersects $y=x+2$ exactly at two points
\item intersects $y=\brak{x+2}^2$ 
\item does NOT intersect $y=\brak{x+3}^2$
\end{enumerate}
\item Let $f:[0,\infty)\rightarrow \Re$ be a continuous \\function such that
\begin{align}
f\brak{x} = 1-2x+ \int_0^x e^{x-t} f(t) dt
\end{align} for all $x\in[0,\infty)$. Then, which of the following statement(s) is (are) TRUE?
\hfill{\brak{JEE Adv.2018}}
\begin{enumerate}
\item The curve $y=f\brak{x}$ passes through the \\point $\brak{1,2}$
\item The curve $y=f\brak{x}$ passes through the \\point $\brak{2,-1}$
\item The area of the region 
\begin{align}
\cbrak{\brak{x,y} \in \sbrak{0,1}\times\Re:f\brak{x}\leq y \leq \sqrt{1-x^2}} 
\end{align}
$$\text{is} \;\frac{\pi-2}{4}$$
\item The area of the region 
\begin{align}
\cbrak{\brak{x,y} \in \sbrak{0,1}\times\Re:f\brak{x}\leq y \leq \sqrt{1-x^2}}
\end{align} 
$$\text{is}\; \frac{\pi-1}{4}$$
\end{enumerate}
\item Let $\Gamma$ denote a curve $y=y\brak{x}$ which is in the first quadrant and let the point $\brak{1,0}$ lie on it. Let the tangent to $\Gamma$ at a point $P$ intersect the $y-axis$ at $Y_p$. If $PY_p$ has length 1 for each point $P$ on $\Gamma$, then which of the following options is/are correct?
\hfill{\brak{JEE Adv.2019}}
\begin{enumerate}
\item $y=-\log_e{\brak{\frac{1+\sqrt{1-x^2}}{x}}} + \sqrt{1-x^2}$
\item $xy^\prime-\sqrt{1-x^2} = 0$
\item $y=\log_e{\brak{\frac{1+\sqrt{1-x^2}}{x}}}-\sqrt{1-x^2}$
\item $xy^\prime+\sqrt{1-x^2} = 0$
\end{enumerate}
\end{enumerate}
\section{E: Subjective Problems}
\begin{enumerate}
\item If $\brak{a+bx}e^{\frac{y}{x}}=x$, then prove that 
\begin{align}
x^3\frac{d^2y}{dx^2}=\brak{x\frac{dy}{dx}-y}^2
\end{align}
\hfill{\brak{1983-3 Marks}}
\item A normal is drawn at a point $P\brak{x,y}$ of a curve. It meets the $x-axis$ at $Q$. If $PQ$ is of constant length $k$, then show that the differential equation describing such curves is
\begin{align}
y\frac{dy}{dx}=\pm \sqrt{k^2-y^2}
\end{align}
Find the equation of such a curve passing through $\brak{0,k}$.
\hfill{\brak{1994-5 Marks}}
\item Let $y=f\brak{x}$ be a curve passing through \brak{1,1} such that the triangle formed by the coordinate axes and the tangent at any point of the curve lies in the first quadrant and has area 2. Form the differential equation and determine all such possible curves.
\hfill{\brak{1995-5 Marks}}
\item Determine the equation of the curve passing through the origin, in the form $y=f\brak{x}$, which satisfies the differential equation 
\begin{align}
\frac{dy}{dx}=\sin\brak{10x+6y}
\end{align}
\hfill{\brak{1996-5 Marks}}
\item Let $u\brak{x}$ and $v\brak{x}$ satisfy the differential equation 
\begin{align}
\frac{du}{dx}+p\brak{x}u=f\brak{x} \;\text{and}
\end{align} 
\begin{align}
\frac{dv}{dx}+p\brak{x}v=g\brak{x},
\end{align} where $p\brak{x} f\brak{x}$ and $g\brak{x}$ are continuous functions. If $u\brak{x_1}>v\brak{x_1}$ for some $x_1$ and $f\brak{x}>g\brak{x}$ for all $x>x_1$, prove that any point $\brak{x,y}$ where $x>x_1$, does not satisfy the equations $y=u\brak{x}$ and $y=v\brak{x}$.
\hfill{\brak{1997-5 Marks}}
\item A curve passing through the point \brak{1,1} has the property that the perpendicular distance of the origin from the normal at any point $P$ of the curve is equal to the distance of $P$ from the $x-axis$. Determine the equation of the curve.
\hfill{\brak{1999-10 Marks}}
\item A country has a food deficit of 10\%. Its population grows continuously at a rate of 3\% per year. Its annual food production every year is 4\% more than that of last year. Assuming that the average food requirement per person remains constant, prove that the country will become self-sufficient in food after $n$ years, where $n$ is the smallest integer bigger than or equal to 
\begin{align}
\frac{\ln10-\ln9}{\ln\brak{1.04}-0.03}
\end{align}
\hfill{\brak{2000-10 Marks}}
\item A hemispherical tank of radius 2 metres is initially full of water and has an outlet of $12cm^2$ cross-sectional area at the bottom. The outlet is opened at some instant. The flow through the outlet is according to the law 
\begin{align}
v\brak{t}=0.6\sqrt{2gh\brak{t}},
\end{align} where $v\brak{t}$ and $h\brak{t}$ are respectively the velocity of the flow through the outlet and the height of water level above the outlet at time $t$, and $g$ is the acceleration due to gravity. Find the time it takes to empty the tank.(Hint: Form a differential equation by relating the decrease of water level to the outflow).
\hfill{\brak{2001-10 Marks}}
\item A right circular cone with radius $R$ and height $H$ contains a liquid which evaporates at a rate proportional to its surface area in contact with air (proportionality constant $=k>0$). Find the time after which the cone is empty.
\hfill{\brak{2003-4 Marks}}
\end{enumerate}
\end{document}
