%iffalse
\let\negmedspace\undefined
\let\negthickspace\undefined
\documentclass[journal,12pt,onecolumn]{IEEEtran}
\usepackage{cite}
\usepackage{amsmath,amssymb,amsfonts,amsthm}
\usepackage{algorithmic}
\usepackage{graphicx}
\usepackage{textcomp}
\usepackage{xcolor}
\usepackage{txfonts}
\usepackage{listings}
\usepackage{enumitem}
\usepackage{mathtools}
\usepackage{gensymb}
\usepackage{comment}
\usepackage[breaklinks=true]{hyperref}
\usepackage{tkz-euclide} 
\usepackage{listings}
\usepackage{gvv}
\usepackage{circuitikz}
%\def\inputGnumericTable{}                         
\usepackage[latin1]{inputenc}                                
\usepackage{color}                                            
\usepackage{array}                                            
\usepackage{longtable}                                       
\usepackage{calc}                                             
\usepackage{multirow}                                         
\usepackage{hhline}                                           
\usepackage{ifthen}                                           
\usepackage{lscape}
\usepackage{tabularx}
\usepackage{array}
\usepackage{float}
\usepackage{multicol}

\newtheorem{theorem}{Theorem}[section]
\newtheorem{problem}{Problem}
\newtheorem{proposition}{Proposition}[section]
\newtheorem{lemma}{Lemma}[section]
\newtheorem{corollary}[theorem]{Corollary}
\newtheorem{example}{Example}[section]
\newtheorem{definition}[problem]{Definition}
\newcommand{\BEQA}{\begin{eqnarray}}
\newcommand{\EEQA}{\end{eqnarray}}
\newcommand{\define}{\stackrel{\triangle}{=}}
\theoremstyle{remark}
\newtheorem{rem}{Remark}

\begin{document}
\bibliographystyle{IEEEtran}
\title{2011-ME-27-39}
\author{EE24BTECH11003 - Akshara Sarma Chennubhatla}
\maketitle
\begin{enumerate}

\item  Figure shows the schematic for the measurement of velocity of air $\brak{\text{density}=1.2 \text{kg}/m^3}$ through a constant area duct using a pitot tube and a water-tube manometer. The differential head of water $\brak{\text{density}=1000 \text{kg}/m^3}$ in the two columns of the manometer is $10$ mm. Take acceleration due to gravity as $9.8$ m/$s^2$. The velocity of air in m/s is
\begin{circuitikz}
\tikzstyle{every node}=[font=\large]
\draw  (4,15) rectangle (8,14.25);
\draw  (4,15) -- (8,15) -- (8.5,15.25) -- (4.5,15.25) -- cycle;
\draw [short] (8,14.25) -- (8.5,14.5);
\draw [short] (8.5,15.25) -- (8.5,14.5);
\draw [short] (4,14.5) -- (3.75,14.75);
\draw [short] (4,14.75) -- (3.75,15);
\draw [short] (4,14.5) -- (3.75,14.75);
\draw [short] (4,14.25) -- (3.75,14.5);
\draw [short] (4,15) -- (3.75,15.25);
\draw  (11.75,14.5) rectangle (15.75,15.25);
\draw  (11.75,15.25) -- (15.75,15.25) -- (16.25,15.5) -- (12.25,15.5) -- cycle;
\draw [short] (15.75,14.5) -- (16.25,14.75);
\draw [short] (16.25,15.5) -- (16.25,14.75);
\draw [->, >=Stealth] (8.25,14.75) -- (9.75,14.75);
\draw [->, >=Stealth] (17.5,15) -- (16,15);
\draw [short] (11.75,14.5) -- (11.5,14.75);
\draw [short] (11.75,14.75) -- (11.5,15);
\draw [short] (11.75,15) -- (11.5,15.25);
\draw [short] (11.75,15.25) -- (11.5,15.5);
\draw [short] (4,12) -- (8.5,12);
\draw [short] (4,11.25) -- (8.5,11.25);
\draw [short] (8.5,12) .. controls (8.5,12) and (8.25,11.5) .. (8.5,11.25);
\draw [short] (4,12) .. controls (3.75,11.75) and (3.75,11.25) .. (4,11.25);
\draw  (12,12) rectangle (16,11.25);
\draw  (12,12) -- (16,12) -- (16.5,12.25) -- (12.5,12.25) -- cycle;
\draw [short] (16,11.25) -- (16.5,11.5);
\draw [short] (16.5,12.25) -- (16.5,11.5);
\draw [short] (12,11.25) -- (11.75,11.5);
\draw [short] (12,11.5) -- (11.75,11.75);
\draw [short] (12,11.75) -- (11.75,12);
\draw [short] (12,12) -- (11.75,12.25);
\draw [->, >=Stealth] (16.5,11) .. controls (17.25,11.5) and (17,12.5) .. (16.25,12.5) ;
\draw [->, >=Stealth] (8.25,11) .. controls (9.5,11) and (8.25,13.25) .. (8,11.75) ;
\draw [short] (8.5,12) .. controls (8.75,11.75) and (8.75,11.25) .. (8.5,11.25);
\draw [short] (4,11.75) -- (3.75,12);
\draw [short] (4,11.5) -- (3.75,11.75);
\draw [short] (4,11.25) -- (3.75,11.5);
\node [font=\Large] at (6.25,15.75) {Tensile load};
\node [font=\large] at (9.25,14) {80 kN};
\node [font=\Large] at (14.25,16) {Compressive load};
\node [font=\large] at (17,14.25) {80 kN};
\node [font=\large] at (9.75,11.75) {64$\pi$ Nm};
\node [font=\large] at (17.75,12.5) {320 Nm};
\node [font=\Large] at (14.25,12.75) {Bending load};
\node [font=\Large] at (6.25,12.5) {Torsional load};
\end{circuitikz}

\hfill{\brak{2011}}
\begin{enumerate}
\item $6.4$
\item $9.0$
\item $12.8$
\item $25.6$
\end{enumerate}

\item The values of enthalpy of steam at the inlet and outlet of a steam turbine in a Rankine cycle are $2800$ kJ/kg and $1800$ kJ/kg respectively. Neglecting pump work, the specific steam consumption in kg/kW-hour is
\hfill{\brak{2011}}
\begin{enumerate}
\item $3.6$
\item $0.36$
\item $0.06$
\item $0.01$
\end{enumerate}

\item The integral $\int_{1}^{3} {\frac{1}{x} dx}$, when evaluated by using Simpson's $\frac{1}{3}$ rule on two equal subintervals each of length $1$, equals
\hfill{\brak{2011}}
\begin{enumerate}
\item $1.000$
\item $1.098$
\item $1.111$
\item $1.120$
\end{enumerate}

\item Two identicals ball bearings $P$ and $Q$ are operating at loads $30$ kN and $45$ kN respectively. The ratio of the life of bearing $P$ to the life of bearing $Q$ is
\hfill{\brak{2011}}
\begin{enumerate}
\item $\frac{81}{16}$
\item $\frac{27}{8}$
\item $\frac{9}{4}$
\item $\frac{3}{2}$
\end{enumerate}

\item For the four-bar linkage shown in the figure, the angular velocity of link $AB$ is $1$ rad/s. The length of link $CD$ is $1.5$ times the length of link $AB$. In the configuration shown, the angular velocity of link $CD$ in rad/s is
\begin{circuitikz}
\tikzstyle{every node}=[font=\large]
\draw [short] (11,18) -- (11,8.25);
\draw [->, >=Stealth] (9.25,10.25) -- (20.75,10.25);
\draw [->, >=Stealth] (9.25,14.75) -- (20.75,14.75);
\draw [line width=1.5pt, short] (11,10.25) -- (13,10.25);
\draw [line width=1.5pt, short] (13,10.25) -- (15,12.75);
\draw [line width=1.5pt, short] (15,12.75) -- (15,11.5);
\draw [line width=1.5pt, short] (15,11.5) -- (20.25,11.5);
\draw [line width=1.5pt, short] (11,17) -- (15,17);
\draw [line width=1.5pt, short] (15,17) -- (17,14.75);
\draw [line width=1.5pt, short] (17,14.75) -- (20.25,14.75);
\draw [dashed] (13,8) -- (13,18.75);
\draw [dashed] (15,18.75) -- (15,8);
\draw [dashed] (17,18.75) -- (17,8);
\draw [->, >=Stealth] (11.5,14.75) -- (11.5,17);
\draw [->, >=Stealth] (10.75,15.25) -- (10.75,16.5);
\draw [->, >=Stealth] (10.75,10.75) -- (10.75,12.5);
\draw [->, >=Stealth] (15.5,11.5) -- (15.5,12.75);
\draw [->, >=Stealth] (17.5,10.25) -- (17.5,11.5);
\draw [short] (15.25,12.75) -- (15.75,12.75);
\draw [<->, >=Stealth] (13,8.5) -- (15,8.5);
\draw [<->, >=Stealth] (15,8.5) -- (17,8.5);
\node [font=\large] at (10.25,11.25) {$i_s$};
\node [font=\large] at (10.25,15.75) {$v_s$};
\node [font=\large] at (12.25,15.75) {600 V};
\node [font=\large] at (16,12.25) {50 A};
\node [font=\large] at (18.25,10.75) {100 A};
\node [font=\large] at (14,7.75) {T1 = 1\mus};
\node [font=\large] at (16,7.75) {T2 = 1\mus};
\node [font=\large] at (21,14.75) {t};
\node [font=\large] at (21,10.25) {t};
\end{circuitikz}

\hfill{\brak{2011}}
\begin{enumerate}
\item $3$
\item $\frac{3}{2}$
\item $1$
\item $\frac{2}{3}$
\end{enumerate}

\item A stone with mass of $0.1$ kg is catapulted as shown in the figure. The total force $F_x \brak{\text{in N}}$ exerted by the rubber band as a function of distance $x \brak{\text{in m}}$ is given by $F_x = 300x^2$. If the stone is displaced by $0.1$ m from the un-stretched position $\brak{x=0}$ of the rubber band, the energy stored in the rubber band is
\begin{center}
\begin{circuitikz}
\tikzstyle{every node}=[font=\LARGE]
\draw [short] (12.25,15.75) .. controls (12,6.5) and (19,9.25) .. (19.5,15);
\draw [short] (12.75,15.75) .. controls (13,6.25) and (19,11.25) .. (19,15);
\draw  (12.5,15.75) ellipse (0.3cm and 0.2cm);
\draw  (19.25,15) ellipse (0.3cm and 0.2cm);
\draw [line width=1.3pt, short] (19.55,14.9) -- (9.5,10.75);
\draw [line width=1.3pt, short] (12.25,15.75) -- (9,11.25);
\draw [ line width=2pt , rotate around={-288:(9.25,11)}] (9.25,11) ellipse (0.25cm and 0.5cm);
\draw [line width=1pt, dashed] (12.5,15.75) -- (19.25,15);
\draw [line width=1.2pt, dashed] (15.5,16.25) -- (15,6.75);
\draw [dashed] (8.5,10.8) -- (15.5,15.5);
\draw [short] (13.25,10.75) .. controls (14.25,9) and (14.25,9.25) .. (14.25,7);
\draw [short] (17,10.3) .. controls (15.75,9) and (15.75,9) .. (15.5,7);
\draw [short] (14.25,7) .. controls (15,6.5) and (15,6.75) .. (15.5,7);
\draw [->, >=Stealth] (15.25,15.5) -- (9.25,11.5);
\draw [->, >=Stealth] (9.5,11.25) -- (11,12.25);
\node [font=\large] at (11,11.75) {$F_x$};
\node [font=\Large] at (11.25,13.25) {x};
\draw [->, >=Stealth] (10.75,10.25) -- (9.25,11);
\node [font=\large] at (11.25,10) {Stone of mass 0.1 kg};
\draw [short] (9.25,11) -- (9.25,12);
\end{circuitikz}
\end{center}

\hfill{\brak{2011}}
\begin{enumerate}
\item $0.01$ J
\item $0.1$ J
\item $1$ J
\item $10$ J
\end{enumerate}

\item Consider the differential equation $\frac{dy}{dx} = \brak{1 + y^2}x$. The general solution with constant $c$ is
\hfill{\brak{2011}}
\begin{enumerate}
\item $y = \tan \frac{x^2}{2} + \tan c$
\item $y = \tan^2 \brak{\frac{x}{2} + c}$
\item $y = \tan^2 \brak{\frac{x}{2}} + c$
\item $y = \tan \brak{\frac{x^2}{2} + c}$
\end{enumerate}

\item An unbiased coin is tossed five times. The outcome of each toss is either a head or a tail. The probability of getting at least one head is
\hfill{\brak{2011}}
\begin{enumerate}
\item $\frac{1}{32}$
\item $\frac{13}{32}$
\item $\frac{16}{32}$
\item $\frac{31}{32}$
\end{enumerate}

\item A mass of $1$ kg is attached to two identical springs each with stiffness $k=20$ kN/m as shown in the figure. Under frictionless condition, the natural frequency of the system in Hz is close to
\begin{circuitikz}
\tikzstyle{every node}=[font=\large]
\draw [short] (9.5,15.5) -- (13.75,15.5);
\draw [short] (9.75,15.5) -- (10.25,16);
\draw [short] (10.25,15.5) -- (10.75,16);
\draw [short] (10.75,15.5) -- (11.25,16);
\draw [short] (11.25,15.5) -- (11.75,16);
\draw [short] (11.75,15.5) -- (12.25,16);
\draw [short] (12.25,15.5) -- (12.75,16);
\draw [short] (12.75,15.5) -- (13.25,16);
\draw [short] (13.25,15.5) -- (13.75,16);
\draw [short] (11.75,15.5) -- (13.75,15.5);
\draw [short] (10,15.5) -- (10.75,12);
\draw [short] (13,15.5) -- (12.25,12);
\draw [short] (10.75,12) -- (12.25,12);
\draw [short] (10.75,12) -- (10.75,9.5);
\draw [short] (12.25,12) -- (12.25,9.5);
\draw [short] (10.75,9.5) -- (12.25,9.5);
\draw [<->, >=Stealth] (8.75,15.5) -- (8.75,11.75);
\draw [<->, >=Stealth] (8.75,11.75) -- (8.75,9.5);
\draw [line width=1.6pt, ->, >=Stealth] (11.5,9.5) -- (11.5,8.25);
\draw [<->, >=Stealth] (10,16.75) -- (13,16.75);
\draw [<->, >=Stealth] (10.75,11.5) -- (12.25,11.5);
\draw [short] (8.5,11.75) -- (9,11.75);
\draw [short] (8.5,15.5) -- (9,15.5);
\draw [short] (8.5,9.5) -- (9,9.5);
\draw [short] (10,17) -- (10,16.25);
\draw [short] (13,17) -- (13,16.25);
\node [font=\large] at (11.5,17.25) {$d_1 = 20 mm$};
\node [font=\large] at (9.25,13.5) {2 m};
\node [font=\large] at (9.5,10.5) {1.5 m};
\node [font=\large] at (13.5,11.5) {$d_2 = 10 mm$};
\node [font=\large] at (12.5,9.5) {A};
\node [font=\large] at (11.75,7.75) {P = 30$\pi$ kN};
\end{circuitikz}

\hfill{\brak{2011}}
\begin{enumerate}
\item $32$
\item $23$
\item $16$
\item $11$
\end{enumerate}

\item The shear strength of a sheet metal is $300$ MPa. The blanking force required to produce a blank of $100$mm diameter from a $1.5$ mm thick sheet is close to
\hfill{\brak{2011}}
\begin{enumerate}
\item $45$ kN
\item $70$ kN
\item $141$ kN
\item $3500$ kN
\end{enumerate}

\item The ratios of the laminar hydrodynamic boundary layer thickness to thermal boundary layer thickness of flows of two fluids $P$ and $Q$ on a flat plate are $\frac{1}{2}$ and $2$ respectively. The Reynolds number based on the plate length for both the4 flows is $10^4$. The Prandtl and Nusselt numbers for $P$ are $\frac{1}{8}$ and $35$ respectively. The Prandtl and Nusselt numbers for $Q$ are respectively
\hfill{\brak{2011}}
\begin{enumerate}
\item $8$ and $140$
\item $8$ and $70$
\item $4$ and $70$
\item $4$ and $35$
\end{enumerate}

\item The crank radius of a single-cylinder I.C engine is $60$ mm and the diameter of the cylinder is $80$ mm. The swept volume of the cylinder in ${cm}^3$ is
\hfill{\brak{2011}}
\begin{enumerate}
\item $48$
\item $96$
\item $302$
\item $603$
\end{enumerate}

\item A pump handling a liquid raises its pressure from $1$ bar to $30$ bar. Take the density of the liquid as $990 \text{kg}/{m}^3$. The isentropic specific work donw by the pump in kJ/kg is
\hfill{\brak{2011}}
\begin{enumerate}
\item $0.10$
\item $0.30$
\item $2.50$
\item $2.93$
\end{enumerate}



\end{enumerate}
\end{document}
