%iffalse
\let\negmedspace\undefined
\let\negthickspace\undefined
\documentclass[journal,12pt,onecolumn]{IEEEtran}
\usepackage{cite}
\usepackage{amsmath,amssymb,amsfonts,amsthm}
\usepackage{algorithmic}
\usepackage{graphicx}
\usepackage{textcomp}
\usepackage{xcolor}
\usepackage{txfonts}
\usepackage{listings}
\usepackage{enumitem}
\usepackage{mathtools}
\usepackage{gensymb}
\usepackage{comment}
\usepackage[breaklinks=true]{hyperref}
\usepackage{tkz-euclide} 
\usepackage{listings}
\usepackage{gvv}
\usepackage{circuitikz}
%\def\inputGnumericTable{}                         
\usepackage[latin1]{inputenc}                                
\usepackage{color}                                            
\usepackage{array}                                            
\usepackage{longtable}                                       
\usepackage{calc}                                             
\usepackage{multirow}                                         
\usepackage{hhline}                                           
\usepackage{ifthen}                                           
\usepackage{lscape}
\usepackage{tabularx}
\usepackage{array}
\usepackage{float}
\usepackage{multicol}

\newtheorem{theorem}{Theorem}[section]
\newtheorem{problem}{Problem}
\newtheorem{proposition}{Proposition}[section]
\newtheorem{lemma}{Lemma}[section]
\newtheorem{corollary}[theorem]{Corollary}
\newtheorem{example}{Example}[section]
\newtheorem{definition}[problem]{Definition}
\newcommand{\BEQA}{\begin{eqnarray}}
\newcommand{\EEQA}{\end{eqnarray}}
\newcommand{\define}{\stackrel{\triangle}{=}}
\theoremstyle{remark}
\newtheorem{rem}{Remark}

\begin{document}
\bibliographystyle{IEEEtran}
\title{2013-XE-27-39}
\author{EE24BTECH11003 - Akshara Sarma Chennubhatla}
\maketitle
\begin{enumerate}[start=27]

\item Flow past a circular cylinder can be produced by superposition of the following elementary potential flows:
\hfill{\brak{2013}}
\begin{enumerate}
\item Uniform flow, doublet
\item Uniform flow, vortex
\item Source, vortex
\item Sink, vortex
\end{enumerate}

\item Let $\delta$, $\delta_1$ and $\delta_2$ denote respectively the boundary-layer thickness, displacement thickness and the momentum thickness for laminar boundary layer flow of an incompressible fluid over a flat plate. The correct relation among these quantities is
\hfill{\brak{2013}}
\begin{enumerate}
\item $\delta < \delta_1 < \delta_2$
\item $\delta > \delta_1 > \delta_2$
\item $\delta > \delta_1 < \delta_2$
\item $\delta < \delta_1 > \delta_2$
\end{enumerate}

\item In the hydrodynamic entry region of a circular duct, the pressure forces balance the sum of
\hfill{\brak{2013}}
\begin{enumerate}
\item viscous and buoyancy forces
\item inertia and buoyancy forces
\item inertia and surface tension forces
\item inertia and viscous forces
\end{enumerate}

\item Bodies with various cross-sectional shapes subjected to cross-flow of air are shown in the following figures. The characteristic dimension of all the shapes is the same. The cross-sectional shape with the largest coefficient of drag $\brak{\text{i.e. sum of the pressure and skin-friction drags}}$, at any moderately large Reynolds number, is
\hfill{\brak{2013}}
\begin{enumerate}
\item 
\item \begin{circuitikz}
\tikzstyle{every node}=[font=\large]
\draw  (4,15) rectangle (8,14.25);
\draw  (4,15) -- (8,15) -- (8.5,15.25) -- (4.5,15.25) -- cycle;
\draw [short] (8,14.25) -- (8.5,14.5);
\draw [short] (8.5,15.25) -- (8.5,14.5);
\draw [short] (4,14.5) -- (3.75,14.75);
\draw [short] (4,14.75) -- (3.75,15);
\draw [short] (4,14.5) -- (3.75,14.75);
\draw [short] (4,14.25) -- (3.75,14.5);
\draw [short] (4,15) -- (3.75,15.25);
\draw  (11.75,14.5) rectangle (15.75,15.25);
\draw  (11.75,15.25) -- (15.75,15.25) -- (16.25,15.5) -- (12.25,15.5) -- cycle;
\draw [short] (15.75,14.5) -- (16.25,14.75);
\draw [short] (16.25,15.5) -- (16.25,14.75);
\draw [->, >=Stealth] (8.25,14.75) -- (9.75,14.75);
\draw [->, >=Stealth] (17.5,15) -- (16,15);
\draw [short] (11.75,14.5) -- (11.5,14.75);
\draw [short] (11.75,14.75) -- (11.5,15);
\draw [short] (11.75,15) -- (11.5,15.25);
\draw [short] (11.75,15.25) -- (11.5,15.5);
\draw [short] (4,12) -- (8.5,12);
\draw [short] (4,11.25) -- (8.5,11.25);
\draw [short] (8.5,12) .. controls (8.5,12) and (8.25,11.5) .. (8.5,11.25);
\draw [short] (4,12) .. controls (3.75,11.75) and (3.75,11.25) .. (4,11.25);
\draw  (12,12) rectangle (16,11.25);
\draw  (12,12) -- (16,12) -- (16.5,12.25) -- (12.5,12.25) -- cycle;
\draw [short] (16,11.25) -- (16.5,11.5);
\draw [short] (16.5,12.25) -- (16.5,11.5);
\draw [short] (12,11.25) -- (11.75,11.5);
\draw [short] (12,11.5) -- (11.75,11.75);
\draw [short] (12,11.75) -- (11.75,12);
\draw [short] (12,12) -- (11.75,12.25);
\draw [->, >=Stealth] (16.5,11) .. controls (17.25,11.5) and (17,12.5) .. (16.25,12.5) ;
\draw [->, >=Stealth] (8.25,11) .. controls (9.5,11) and (8.25,13.25) .. (8,11.75) ;
\draw [short] (8.5,12) .. controls (8.75,11.75) and (8.75,11.25) .. (8.5,11.25);
\draw [short] (4,11.75) -- (3.75,12);
\draw [short] (4,11.5) -- (3.75,11.75);
\draw [short] (4,11.25) -- (3.75,11.5);
\node [font=\Large] at (6.25,15.75) {Tensile load};
\node [font=\large] at (9.25,14) {80 kN};
\node [font=\Large] at (14.25,16) {Compressive load};
\node [font=\large] at (17,14.25) {80 kN};
\node [font=\large] at (9.75,11.75) {64$\pi$ Nm};
\node [font=\large] at (17.75,12.5) {320 Nm};
\node [font=\Large] at (14.25,12.75) {Bending load};
\node [font=\Large] at (6.25,12.5) {Torsional load};
\end{circuitikz}

\item \begin{circuitikz}
\tikzstyle{every node}=[font=\large]
\draw [short] (11,18) -- (11,8.25);
\draw [->, >=Stealth] (9.25,10.25) -- (20.75,10.25);
\draw [->, >=Stealth] (9.25,14.75) -- (20.75,14.75);
\draw [line width=1.5pt, short] (11,10.25) -- (13,10.25);
\draw [line width=1.5pt, short] (13,10.25) -- (15,12.75);
\draw [line width=1.5pt, short] (15,12.75) -- (15,11.5);
\draw [line width=1.5pt, short] (15,11.5) -- (20.25,11.5);
\draw [line width=1.5pt, short] (11,17) -- (15,17);
\draw [line width=1.5pt, short] (15,17) -- (17,14.75);
\draw [line width=1.5pt, short] (17,14.75) -- (20.25,14.75);
\draw [dashed] (13,8) -- (13,18.75);
\draw [dashed] (15,18.75) -- (15,8);
\draw [dashed] (17,18.75) -- (17,8);
\draw [->, >=Stealth] (11.5,14.75) -- (11.5,17);
\draw [->, >=Stealth] (10.75,15.25) -- (10.75,16.5);
\draw [->, >=Stealth] (10.75,10.75) -- (10.75,12.5);
\draw [->, >=Stealth] (15.5,11.5) -- (15.5,12.75);
\draw [->, >=Stealth] (17.5,10.25) -- (17.5,11.5);
\draw [short] (15.25,12.75) -- (15.75,12.75);
\draw [<->, >=Stealth] (13,8.5) -- (15,8.5);
\draw [<->, >=Stealth] (15,8.5) -- (17,8.5);
\node [font=\large] at (10.25,11.25) {$i_s$};
\node [font=\large] at (10.25,15.75) {$v_s$};
\node [font=\large] at (12.25,15.75) {600 V};
\node [font=\large] at (16,12.25) {50 A};
\node [font=\large] at (18.25,10.75) {100 A};
\node [font=\large] at (14,7.75) {T1 = 1\mus};
\node [font=\large] at (16,7.75) {T2 = 1\mus};
\node [font=\large] at (21,14.75) {t};
\node [font=\large] at (21,10.25) {t};
\end{circuitikz}

\item \begin{center}
\begin{circuitikz}
\tikzstyle{every node}=[font=\LARGE]
\draw [short] (12.25,15.75) .. controls (12,6.5) and (19,9.25) .. (19.5,15);
\draw [short] (12.75,15.75) .. controls (13,6.25) and (19,11.25) .. (19,15);
\draw  (12.5,15.75) ellipse (0.3cm and 0.2cm);
\draw  (19.25,15) ellipse (0.3cm and 0.2cm);
\draw [line width=1.3pt, short] (19.55,14.9) -- (9.5,10.75);
\draw [line width=1.3pt, short] (12.25,15.75) -- (9,11.25);
\draw [ line width=2pt , rotate around={-288:(9.25,11)}] (9.25,11) ellipse (0.25cm and 0.5cm);
\draw [line width=1pt, dashed] (12.5,15.75) -- (19.25,15);
\draw [line width=1.2pt, dashed] (15.5,16.25) -- (15,6.75);
\draw [dashed] (8.5,10.8) -- (15.5,15.5);
\draw [short] (13.25,10.75) .. controls (14.25,9) and (14.25,9.25) .. (14.25,7);
\draw [short] (17,10.3) .. controls (15.75,9) and (15.75,9) .. (15.5,7);
\draw [short] (14.25,7) .. controls (15,6.5) and (15,6.75) .. (15.5,7);
\draw [->, >=Stealth] (15.25,15.5) -- (9.25,11.5);
\draw [->, >=Stealth] (9.5,11.25) -- (11,12.25);
\node [font=\large] at (11,11.75) {$F_x$};
\node [font=\Large] at (11.25,13.25) {x};
\draw [->, >=Stealth] (10.75,10.25) -- (9.25,11);
\node [font=\large] at (11.25,10) {Stone of mass 0.1 kg};
\draw [short] (9.25,11) -- (9.25,12);
\end{circuitikz}
\end{center}
\\
\begin{circuitikz}
\tikzstyle{every node}=[font=\large]
\draw [short] (9.5,15.5) -- (13.75,15.5);
\draw [short] (9.75,15.5) -- (10.25,16);
\draw [short] (10.25,15.5) -- (10.75,16);
\draw [short] (10.75,15.5) -- (11.25,16);
\draw [short] (11.25,15.5) -- (11.75,16);
\draw [short] (11.75,15.5) -- (12.25,16);
\draw [short] (12.25,15.5) -- (12.75,16);
\draw [short] (12.75,15.5) -- (13.25,16);
\draw [short] (13.25,15.5) -- (13.75,16);
\draw [short] (11.75,15.5) -- (13.75,15.5);
\draw [short] (10,15.5) -- (10.75,12);
\draw [short] (13,15.5) -- (12.25,12);
\draw [short] (10.75,12) -- (12.25,12);
\draw [short] (10.75,12) -- (10.75,9.5);
\draw [short] (12.25,12) -- (12.25,9.5);
\draw [short] (10.75,9.5) -- (12.25,9.5);
\draw [<->, >=Stealth] (8.75,15.5) -- (8.75,11.75);
\draw [<->, >=Stealth] (8.75,11.75) -- (8.75,9.5);
\draw [line width=1.6pt, ->, >=Stealth] (11.5,9.5) -- (11.5,8.25);
\draw [<->, >=Stealth] (10,16.75) -- (13,16.75);
\draw [<->, >=Stealth] (10.75,11.5) -- (12.25,11.5);
\draw [short] (8.5,11.75) -- (9,11.75);
\draw [short] (8.5,15.5) -- (9,15.5);
\draw [short] (8.5,9.5) -- (9,9.5);
\draw [short] (10,17) -- (10,16.25);
\draw [short] (13,17) -- (13,16.25);
\node [font=\large] at (11.5,17.25) {$d_1 = 20 mm$};
\node [font=\large] at (9.25,13.5) {2 m};
\node [font=\large] at (9.5,10.5) {1.5 m};
\node [font=\large] at (13.5,11.5) {$d_2 = 10 mm$};
\node [font=\large] at (12.5,9.5) {A};
\node [font=\large] at (11.75,7.75) {P = 30$\pi$ kN};
\end{circuitikz}

\end{enumerate}

\item A U-tube of a very small bore, with its limbs in a vertical plane and filled with a liquid of density $\rho$, up to a height of $h$, is rotated about a vertical axis, with an angular velocity of $\omega$, as shown in the Figure. The radius of each limb from the axis of rotation is $R$. Let $p_a$ be the atmospheric pressure and $g$, the gravitational acceleration. The angular velocity at which the pressure at the point $O$ becomes half of the atmospheric pressure is given by
\begin{center}
\begin{circuitikz}
\tikzstyle{every node}=[font=\normalsize]
\draw [ color={rgb,255:red,242; green,133; blue,0} ] (5,14.75) rectangle (17.5,14.5);
\draw [ color={rgb,255:red,242; green,133; blue,0} ] (6,14.5) rectangle (6.5,14.25);
\draw [ color={rgb,255:red,242; green,133; blue,0} ] (6.25,14.5) rectangle (6.75,14.25);
\draw [ color={rgb,255:red,242; green,133; blue,0} ] (9.75,14.5) rectangle (10.25,14.25);
\draw [ color={rgb,255:red,242; green,133; blue,0} ] (10,14.5) rectangle (10.5,14.25);
\draw [ color={rgb,255:red,0; green,84; blue,194}, short] (5,14.75) -- (5,16);
\draw [ color={rgb,255:red,0; green,84; blue,194}, short] (5,16) -- (5.25,16);
\draw [ color={rgb,255:red,0; green,84; blue,194}, short] (5.25,16) -- (5.5,15);
\draw [ color={rgb,255:red,0; green,84; blue,194}, short] (5.5,15) -- (11.25,15);
\draw [ color={rgb,255:red,0; green,84; blue,194}, short] (11.25,15) -- (11.5,16);
\draw [ color={rgb,255:red,0; green,84; blue,194}, short] (11.5,16) -- (11.75,16);
\draw [ color={rgb,255:red,0; green,84; blue,194}, short] (11.75,16) -- (11.75,14.75);
\draw [ color={rgb,255:red,242; green,133; blue,0} ] (6,15) rectangle (6.75,15.25);
\draw [ color={rgb,255:red,242; green,133; blue,0} ] (6.5,15) rectangle (6.25,15.5);
\draw [ color={rgb,255:red,242; green,133; blue,0} ] (9.75,15) rectangle (10.5,15.25);
\draw [ color={rgb,255:red,242; green,133; blue,0} ] (10,15) rectangle (10.25,15.5);
\draw [ color={rgb,255:red,242; green,133; blue,0} ] (5,12.25) rectangle (17.5,6);
\draw [ color={rgb,255:red,0; green,84; blue,194}, short] (5,12.25) -- (5,13);
\draw [ color={rgb,255:red,0; green,84; blue,194}, short] (11.75,12.25) -- (11.75,13);
\draw [ color={rgb,255:red,0; green,84; blue,194}, dashed] (6.25,15.75) -- (6.25,5.5);
\draw [ color={rgb,255:red,0; green,84; blue,194}, dashed] (10,15.75) -- (10,5.5);
\draw [ color={rgb,255:red,242; green,133; blue,0}, short] (6.25,11.25) -- (6,11.25);
\draw [ color={rgb,255:red,242; green,133; blue,0}, short] (6.25,11.25) -- (6.5,11.25);
\draw [ color={rgb,255:red,242; green,133; blue,0}, short] (6.5,11.25) -- (6.75,11);
\draw [ color={rgb,255:red,242; green,133; blue,0}, short] (6.75,11) -- (6.5,10.75);
\draw [ color={rgb,255:red,242; green,133; blue,0}, short] (6.5,10.75) -- (6,10.75);
\draw [ color={rgb,255:red,242; green,133; blue,0}, short] (6,10.75) -- (5.75,11);
\draw [ color={rgb,255:red,242; green,133; blue,0}, short] (5.75,11) -- (6,11.25);
\draw [ color={rgb,255:red,242; green,133; blue,0}, short] (9.75,11.25) -- (10.25,11.25);
\draw [ color={rgb,255:red,242; green,133; blue,0}, short] (10.25,11.25) -- (10.5,11);
\draw [ color={rgb,255:red,242; green,133; blue,0}, short] (10.5,11) -- (10.25,10.75);
\draw [ color={rgb,255:red,242; green,133; blue,0}, short] (10.25,10.75) -- (9.75,10.75);
\draw [ color={rgb,255:red,242; green,133; blue,0}, short] (9.75,10.75) -- (9.5,11);
\draw [ color={rgb,255:red,242; green,133; blue,0}, short] (9.5,11) -- (9.75,11.25);
\draw [ color={rgb,255:red,242; green,133; blue,0}, short] (6,7.5) -- (6.5,7.5);
\draw [ color={rgb,255:red,242; green,133; blue,0}, short] (6.5,7.5) -- (6.75,7.25);
\draw [ color={rgb,255:red,242; green,133; blue,0}, short] (6.75,7.25) -- (6.5,7);
\draw [ color={rgb,255:red,242; green,133; blue,0}, short] (6.5,7) -- (6,7);
\draw [ color={rgb,255:red,242; green,133; blue,0}, short] (6,7) -- (5.75,7.25);
\draw [ color={rgb,255:red,242; green,133; blue,0}, short] (5.75,7.25) -- (6,7.5);
\draw [ color={rgb,255:red,242; green,133; blue,0}, short] (9.75,7.5) -- (10.25,7.5);
\draw [ color={rgb,255:red,242; green,133; blue,0}, short] (10.25,7.5) -- (10.5,7.25);
\draw [ color={rgb,255:red,242; green,133; blue,0}, short] (10.5,7.25) -- (10.25,7);
\draw [ color={rgb,255:red,242; green,133; blue,0}, short] (10.25,7) -- (9.75,7);
\draw [ color={rgb,255:red,242; green,133; blue,0}, short] (9.75,7) -- (9.5,7.25);
\draw [ color={rgb,255:red,242; green,133; blue,0}, short] (9.5,7.25) -- (9.75,7.5);
\draw [ color={rgb,255:red,0; green,84; blue,194}, dashed] (5.5,11) -- (13.75,11);
\draw [ color={rgb,255:red,0; green,84; blue,194}, dashed] (5.5,7.25) -- (13.75,7.25);
\draw [ color={rgb,255:red,0; green,84; blue,194}, dashed] (6.75,9.25) -- (18,9.25);
\draw [ color={rgb,255:red,0; green,84; blue,194}, dashed] (8.25,5.75) -- (10.25,5.75);
\draw [ color={rgb,255:red,0; green,84; blue,194}, dashed] (9.25,15.25) -- (11.25,15.25);
\draw [short] (5,13) .. controls (5.5,13.25) and (5.75,13) .. (6.25,13);
\draw [short] (6.25,13) .. controls (7,13.25) and (7.25,13.25) .. (8,13.25);
\draw [short] (8,13.25) .. controls (8.75,13) and (9,13) .. (10,13);
\draw [short] (10,13) -- (11.75,13.25);
\draw [short] (5,5.5) .. controls (5.75,5.25) and (5.5,5.25) .. (6.25,5.25);
\draw [short] (6.25,5.25) .. controls (7,5) and (7,5.5) .. (8,5.5);
\draw [short] (8,5.5) -- (10,5.5);
\draw [ color={rgb,255:red,0; green,84; blue,194}, short] (10,5.5) -- (11.25,5.5);
\draw [ color={rgb,255:red,0; green,84; blue, 194}, short] (11.25,5.5) -- (11.25,6);
\draw [ color={rgb,255:red,0; green,84; blue, 194}, short] (5,5.5) -- (5,6);
\draw [ color={rgb,255:red,0; green,84; blue, 194}, short] (11.75,13.25) -- (11.75,13);
\draw [ color={rgb,255:red,0; green,84; blue, 194}, <->, >=Stealth] (5,4.75) -- (8,4.75);
\draw [ color={rgb,255:red,0; green,84; blue, 194}, <->, >=Stealth] (8.25,4.75) -- (11.25,4.75);
\draw [ color={rgb,255:red,0; green,84; blue, 194}, <->, >=Stealth] (11.25,4.75) -- (18.75,4.75);
\draw [ color={rgb,255:red,0; green,84; blue, 194}, <->, >=Stealth] (18,12.25) -- (18,6);
\node [font=\normalsize, color={rgb,255:red,0; green,84; blue, 194}] at (18.25,9.5) {200};
\node [font=\normalsize] at (11,10.25) {50};
\node [font=\normalsize] at (11,8.25) {50};
\node [font=\normalsize] at (5.75,11.5) {A};
\node [font=\normalsize] at (9.5,11.5) {B};
\node [font=\normalsize] at (8.25,9.5) {O};
\node [font=\normalsize] at (5.75,7.75) {C};
\node [font=\normalsize] at (9.5,7.75) {D};
\node [font=\normalsize] at (6.25,4.5) {100};
\node [font=\normalsize] at (9.75,4.5) {100};
\node [font=\normalsize] at (14.5,4.25) {300};
\node [font=\normalsize] at (15.25,13) {F=10kN};
\node [font=\normalsize] at (8.5,16.75) {200};
\node [font=\normalsize] at (12.75,15.75) {Steel channel};
\node [font=\normalsize] at (16.25,16.75) {Not to scale};
\node [font=\normalsize] at (16.25,16.25) {Dimensions in mm};
\draw [ color={rgb,255:red,0; green,84; blue,194}, <->, >=Stealth] (5.25,16.5) -- (11.5,16.5);
\draw [ color={rgb,255:red,0; green,84; blue,194}, ->, >=Stealth] (16.25,14) -- (16.25,12.25);
\draw [ color={rgb,255:red,0; green,84; blue,194}, <->, >=Stealth] (10.75,11) -- (10.75,9.25);
\draw [ color={rgb,255:red,0; green,84; blue,194}, <->, >=Stealth] (10.75,9.25) -- (10.75,7.25);
\end{circuitikz}

\end{center}
\hfill{\brak{2013}}
\begin{enumerate}
\item $\sqrt{\frac{p_a + 2\rho gh}{\rho R^2}}$
\item $\sqrt{\frac{2\brak{p_a + \rho gh}}{\rho R^2}}$
\item $\sqrt{\frac{p_a + 2\rho gh}{2\rho R^2}}$
\item $\sqrt{\frac{p_a + \rho gh}{2\rho R^2}}$
\end{enumerate}

\item An incompressible fluid at a pressure of $150$ kPa $\brak{\text{absolute}}$ flows steadily through a two-dimensional channel with a velocity of $5$ m/s as shown in the Figure. The channel has a $90^{\circ}$ bend. The fluid leaves the channel with a pressure of $100$ kPa $\brak{\text{absolute}}$ and linearly-varying velocity profile. $v_{max}$ is four times $v_{min}$. The density of the fluid is $914.3$ kg/$m^3$. The velocity $v_{min}$, in m/s, is
\begin{circuitikz}
\tikzstyle{every node}=[font=\Large]
\draw [short] (9.5,12.25) -- (13.25,12.25);
\draw [short] (9.5,10) -- (15.25,10);
\draw [short] (13.25,12.25) -- (13.25,16.25);
\draw [short] (15.25,16.25) -- (15.25,10);
\draw [short] (13.25,16.25) -- (15.25,16.25);
\draw [short] (13.25,19.5) -- (15.25,17.5);
\draw [->, >=Stealth] (7.75,12.25) -- (9.5,12.25);
\draw [->, >=Stealth] (7.75,12) -- (9.5,12);
\draw [->, >=Stealth] (7.75,11.75) -- (9.5,11.75);
\draw [->, >=Stealth] (7.75,11.5) -- (9.5,11.5);
\draw [->, >=Stealth] (7.75,11.25) -- (9.5,11.25);
\draw [->, >=Stealth] (7.75,11) -- (9.5,11);
\draw [->, >=Stealth] (7.75,10.75) -- (9.5,10.75);
\draw [->, >=Stealth] (7.75,10.5) -- (9.5,10.5);
\draw [->, >=Stealth] (7.75,10.25) -- (9.5,10.25);
\draw [->, >=Stealth] (7.75,10) -- (9.5,10);
\draw [->, >=Stealth] (13.25,16.25) -- (13.25,19.5);
\draw [->, >=Stealth] (13.75,16.25) -- (13.75,19);
\draw [->, >=Stealth] (15.25,16.25) -- (15.25,17.5);
\draw [->, >=Stealth] (14.75,16.25) -- (14.75,18);
\draw [->, >=Stealth] (14.25,16.25) -- (14.25,18.5);
\draw [short] (7.75,12.25) -- (7.75,10);
\draw [short] (9.5,12.25) -- (9.5,10);
\draw [->, >=Stealth] (15,16.25) -- (15,17.75);
\draw [->, >=Stealth] (14.5,16.25) -- (14.5,18.25);
\draw [->, >=Stealth] (14,16.25) -- (14,18.75);
\draw [->, >=Stealth] (13.5,16.25) -- (13.5,19.25);
\draw [<->, >=Stealth] (13.25,14.25) -- (15.25,14.25);
\draw [<->, >=Stealth] (10.75,12.25) -- (10.75,10);
\node [font=\Large] at (6.5,11.25) {5 m/s};
\node [font=\Large] at (12,11.25) {50 mm};
\node [font=\Large] at (14.25,14.75) {60 mm};
\node [font=\Large] at (12.5,17.75) {$V_{max}$};
\node [font=\Large] at (16,16.75) {$V_{min}$};
\end{circuitikz}

\hfill{\brak{2013}}
\begin{enumerate}
\item $25$
\item $2.5$
\item $2.0$
\item $0.2$
\end{enumerate}

\item The velocity vector corresponding to a flow field is given, with usual notation, by $\vec{V} = 3x\vec{i} + 4xy\vec{j}$. The magnitude of rotation at the point $\brak{2,2}$ in rad/s is
\hfill{\brak{2013}}
\begin{enumerate}
\item $0.75$
\item $1.33$
\item $2$
\item $4$
\end{enumerate}

\item The stream function for a potential flow field is given by $\psi = x^2 - y^2$. The corresponding potential function, assuming zero potential at the origin, is
\hfill{\brak{2013}}
\begin{enumerate}
\item $x^2 + y^2$
\item $2xy$
\item $x^2 - y^2$
\item $x - y$
\end{enumerate}

\item Fully developed flow of an oil takes place in a pipe of inner diameter $50$ mm. The pressure drop per metre length of the pipe is $2$ kPa. Determine the shear stress, in Pa, at the pipe wall.$\rule{2cm}{0.1pt}$
\hfill{\brak{2013}}

\item The Darcy friction factor $f$ for a smooth pipe is given by $f = \frac{64}{Re}$ for laminar flow and by $f = \frac{0.3}{Re^{0.25}}$ for turbulent flow, where $Re$ is the Reynolds number based on the diameter. For fully developed flow of a fluid of density $1000$ kg/$m^3$ and dynamic viscosity $0.001$ Pa.s through a smooth pipe of diameter $10$ mm with a velocity of $1$ m/s, determine the Darcy friction factor.$\rule{2cm}{0.1pt}$
\hfill{\brak{2013}}

\item Air flows steadily through a channel. The stagnation and static pressures at a point in the flow are measured by a Pitot tube and a wall pressure tap, respectively. The pressure difference is found to be $20$ mm Hg. The densities of air, water and mercury, in kg/$m^3$, are $1.18$, $1000$ and $13600$, respectively. The gravitational acceleration is $9.81$ m/$s^2$. Determine the air speed in m/s.$\rule{2cm}{0.1pt}$
\hfill{\brak{2013}}\\

\textbf{Common Data for Questions 12 and 13:}\\

The velocity field within a laminar boundary layer is given by the expression:
\begin{align*}
\vec{V} = \frac{Bu_{\infty}y}{x^{\frac{3}{2}}}\vec{i} + \frac{Bu_{\infty}y^{2}}{4x^{\frac{5}{2}}}\vec{j} 
\end{align*}
where $B = 100m^{\frac{1}{2}}$ and the free stream velocity $u_{\infty} = 0.1$ m/s.

\item Calculate the x-direction component of the acceleration in m/$s^2$ at the point $x = 0.5$ m and $y = 50$ mm.$\rule{2cm}{0.1pt}$
\hfill{\brak{2013}}

\item Find the slope of the streamline passing through the point $x = 0.5$ m and $y = 50$ mm.$\rule{2cm}{0.1pt}$
\hfill{\brak{2013}}


\end{enumerate}
\end{document}

