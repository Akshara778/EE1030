%iffalse
\let\negmedspace\undefined
\let\negthickspace\undefined
\documentclass[journal,12pt,onecolumn]{IEEEtran}
\usepackage{cite}
\usepackage{amsmath,amssymb,amsfonts,amsthm}
\usepackage{algorithmic}
\usepackage{graphicx}
\usepackage{textcomp}
\usepackage{xcolor}
\usepackage{txfonts}
\usepackage{listings}
\usepackage{enumitem}
\usepackage{mathtools}
\usepackage{gensymb}
\usepackage{comment}
\usepackage[breaklinks=true]{hyperref}
\usepackage{tkz-euclide} 
\usepackage{listings}
\usepackage{gvv}
\usepackage{circuitikz}
%\def\inputGnumericTable{}                         
\usepackage[latin1]{inputenc}                                
\usepackage{color}                                            
\usepackage{array}                                            
\usepackage{longtable}                                       
\usepackage{calc}                                             
\usepackage{multirow}                                         
\usepackage{hhline}                                           
\usepackage{ifthen}                                           
\usepackage{lscape}
\usepackage{tabularx}
\usepackage{array}
\usepackage{float}
\usepackage{multicol}

\newtheorem{theorem}{Theorem}[section]
\newtheorem{problem}{Problem}
\newtheorem{proposition}{Proposition}[section]
\newtheorem{lemma}{Lemma}[section]
\newtheorem{corollary}[theorem]{Corollary}
\newtheorem{example}{Example}[section]
\newtheorem{definition}[problem]{Definition}
\newcommand{\BEQA}{\begin{eqnarray}}
\newcommand{\EEQA}{\end{eqnarray}}
\newcommand{\define}{\stackrel{\triangle}{=}}
\theoremstyle{remark}
\newtheorem{rem}{Remark}

\begin{document}
\bibliographystyle{IEEEtran}
\title{2012-AE-14-26}
\author{EE24BTECH11003 - Akshara Sarma Chennubhatla}
\maketitle
\begin{enumerate}[start=14]

\item During the ground roll manoeuvre of an aircraft, the force$\brak{s}$ acting on it parallel to the direction of motion
\hfill{\brak{2012}}
\begin{enumerate}
\item is thrust alone.
\item is drag alone.
\item are both thrust and drag.
\item are thrust, drag and a part of both weight and life.
\end{enumerate}

\item An aircraft in a steady climb suddenly experiences a $10\%$ drop in thrust. After a new equilibrium is reached at the same speed, the new rate of climb is
\hfill{\brak{2012}}
\begin{enumerate}
\item lower by exactly $10\%$
\item lower by more than $10\%$
\item lower by less than $10\%$
\item an unpredictable quantity.
\end{enumerate}

\item In an aircraft, the dive manoeuvre can be initiated by
\hfill{\brak{2012}}
\begin{enumerate}
\item reducing the engine thrust alone.
\item reducing the angle of attack alone.
\item generating a nose down pitch rate.
\item increasing the engine thrust alone.
\end{enumerate}

\item In an aircraft, the elevator control effectiveness determines
\hfill{\brak{2012}}
\begin{enumerate}
\item turn radius.
\item rate of climb.
\item forward-most location of the centre of gravity.
\item aft-most location of the centre of gravity.
\end{enumerate}

\item The Mach angle for a flow at Mach $2.0$ ditribution is
\hfill{\brak{2012}}
\begin{enumerate}
\item $30^\circ$
\item $45^\circ$
\item $60^\circ$
\item $90^\circ$
\end{enumerate}

\item For a wing of aspect ratio $AR$, having an elliptical lift distribution, the induced drag coefficient is $\brak{\text{where }C_L\text{ is the lift coefficient}}$
\hfill{\brak{2012}}
\begin{enumerate}
\item $\frac{C_L}{\pi AR}$
\item $\frac{C_L^{2}}{\pi AR}$
\item $\frac{C_L}{2\pi AR}$
\item $\frac{C_L^{2}}{\pi A R^{2}}$
\end{enumerate}

\item Bernoulli's equation is valid under steady state
\hfill{\brak{2012}}
\begin{enumerate}
\item only along a streamline in inviscid flow, and between any two points in potential flow.
\item between any two points in both inviscid and potential flow.
\item between any two points in inviscid flow, and only along a streamline in potential flow.
\item only along a streamline in both inviscid and potential flow.
\end{enumerate}

\item The ratio of flight speed to the exhaust velocity for maximum propulsion efficiency is
\hfill{\brak{2012}}
\begin{enumerate}
\item $0.0$
\item $0.5$
\item $1.0$
\item $2.0$
\end{enumerate}

\item The ideal static pressure coefficient of a diffuser with an area ratio of $2.0$ is
\hfill{\brak{2012}}
\begin{enumerate}
\item $0.25$
\item $0.50$
\item $0.75$
\item $1.0$
\end{enumerate}

\item A rocket is to be launched from the bottom of a very deep crater on Mars for earth return. The
specific impulse of the rocket, measured in seconds, is to be normalized by the acceleration due to
gravity at
\hfill{\brak{2012}}
\begin{enumerate}
\item the bottom of the crater on Mars.
\item Mars standard "sea level".
\item earth's standard sea level.
\item the same depth of the crater on earth.
\end{enumerate}

\item In a semi-monocoque construction of an aircraft wing, the skin and spar webs are the primary
carriers of
\hfill{\brak{2012}}
\begin{enumerate}
\item shear stresses due to an aerodynamic moment component alone.
\item normal $\brak{\text{bending}}$ stresses due to aerodynamic forces.
\item shear stresses due to aerodynamic forces alone.
\item shear stresses due to aerodynamic forces and a moment component.
\end{enumerate}

\item The logarithmic decrement measured for a viscously damped single degree of freedom system is $0.125$. The value of the damping factor in $\%$ is closest to
\hfill{\brak{2012}}
\begin{enumerate}
\item $0.5$
\item $1.0$
\item $1.5$
\item $2.0$
\end{enumerate}

\item The integration $\int_{0}^{1}x^3 dx$ computed using trapezoidal rule with $n = 4$ intervals is $\rule{2cm}{0.1pt}$.
\hfill{\brak{2012}}

\end{enumerate}
\end{document}
