%iffalse
\let\negmedspace\undefined
\let\negthickspace\undefined
\documentclass[journal,12pt,onecolumn]{IEEEtran}
\usepackage{cite}
\usepackage{amsmath,amssymb,amsfonts,amsthm}
\usepackage{algorithmic}
\usepackage{graphicx}
\usepackage{textcomp}
\usepackage{xcolor}
\usepackage{txfonts}
\usepackage{listings}
\usepackage{enumitem}
\usepackage{mathtools}
\usepackage{gensymb}
\usepackage{comment}
\usepackage[breaklinks=true]{hyperref}
\usepackage{tkz-euclide} 
\usepackage{listings}
\usepackage{gvv}
\usepackage{circuitikz}
%\def\inputGnumericTable{}                         
\usepackage[latin1]{inputenc}                                
\usepackage{color}                                            
\usepackage{array}                                            
\usepackage{longtable}                                       
\usepackage{calc}                                             
\usepackage{multirow}                                         
\usepackage{hhline}                                           
\usepackage{ifthen}                                           
\usepackage{lscape}
\usepackage{tabularx}
\usepackage{array}
\usepackage{float}
\usepackage{multicol}

\newtheorem{theorem}{Theorem}[section]
\newtheorem{problem}{Problem}
\newtheorem{proposition}{Proposition}[section]
\newtheorem{lemma}{Lemma}[section]
\newtheorem{corollary}[theorem]{Corollary}
\newtheorem{example}{Example}[section]
\newtheorem{definition}[problem]{Definition}
\newcommand{\BEQA}{\begin{eqnarray}}
\newcommand{\EEQA}{\end{eqnarray}}
\newcommand{\define}{\stackrel{\triangle}{=}}
\theoremstyle{remark}
\newtheorem{rem}{Remark}

\begin{document}
\bibliographystyle{IEEEtran}
\title{2014-PH-14-26}
\author{EE24BTECH11003 - Akshara Sarma Chennubhatla}
\maketitle
\begin{enumerate}

\item If the half-life of an elementary particle moving with speed $0.9c$ in the laboratory frame is $5 x 10^{-8}$ s, then the proper half-life is $\rule{2cm}{0.1pt} x 10^{-8}$ s. $\brak{c=3 x 10^8 \text{ m/s}}$
\hfill{\brak{2014}}

\item An unpolarized light wave is incident from air on a glass surface at the Brewster angle. The angle between the reflected and the refracted wave is
\hfill{\brak{2014}}
\begin{enumerate}
\item $0^{\circ}$
\item $45^{\circ}$
\item $90^{\circ}$
\item $120^{\circ}$
\end{enumerate}

\item Two masses $m$ and $3m$ are attached to the two ends of a massless spring with force constant $K$. If $m=100$ g and $K=0.3$ N/m, then the natural angular frequency of oscillation is $\rule{2cm}{0.1pt}$ Hz.
\hfill{\brak{2014}}

\item The electric field of a uniform plane wave propagating in a dielectric, non-conducting medium is given by,
\begin{align*}
\vec{E} = \vec{x}10\cos\brak{6\pi x 10^7t - 0.4\pi z}V/m. 
\end{align*}
The phase velocity of the wave is $\rule{2cm}{0.1pt} x 10^8$ m/s.
\hfill{\brak{2014}}

\item The matrix $A = \frac{1}{\sqrt{3}}\myvec{1 & 1+i\\1-i & -1}$ is
\hfill{\brak{2014}}
\begin{enumerate}
\item orthogonal
\item symmetric
\item anti-symmetric
\item unitary
\end{enumerate}

\item The recoil momentum of an atom is $p_A$ when it emits an infrared photon of wavelength $1500$ nm, and it is $p_B$ when it emits a photon of visible wavelength $500$ nm. The ratio $\frac{p_A}{p_B}$ is
\hfill{\brak{2014}}
\begin{enumerate}
\item $1:1$
\item $1:\sqrt{3}$
\item $1:3$
\item $3:2$
\end{enumerate}

\item For a gas under isothermal conditions, its pressure $P$ varies with volume $V$ as $P \propto V^{-\frac{5}{3}}$. The bulk modulus $B$ is proportional to
\hfill{\brak{2014}}
\begin{enumerate}
\item $V^{-\frac{1}{2}}$
\item $V^{-\frac{2}{3}}$
\item $V^{-\frac{3}{5}}$
\item $V^{-\frac{5}{3}}$
\end{enumerate}

\item Which one of the following high energy processes is allowed by conservation laws?
\hfill{\brak{2014}}
\begin{enumerate}
\item $p + \bar{p} \to \Lambda^0 + \Lambda^0$
\item $\pi^- + p \to \pi^0 + n$
\item $n \to p + e^- + v_e$
\item $\mu^+ \to e^+ + \gamma$
\end{enumerate}

\item The length element $ds$ of an arc is given by, $\brak{ds}^2 = 2\brak{dx^2}^2 + \sqrt{3}dx^1dx^2$. The metric tensor $g_g$ is
\hfill{\brak{2014}}
\begin{enumerate}
\item $\myvec{2 & \sqrt{3}\\\sqrt{3} & 1}$
\item $\myvec{2 & \sqrt{\frac{3}{2}}\\\sqrt{\frac{3}{2}} & 1}$
\item $\myvec{2 & 1\\\sqrt{\frac{3}{2}} & \sqrt{\frac{3}{2}}}$
\item $\myvec{1 & \sqrt{\frac{3}{2}}\\\sqrt{\frac{3}{2}} & 2}$
\end{enumerate}

\item The ground state and the first excited state wave functions of a one dimensional infinite potential well are $\psi_1$ and $\psi_2$, respectively. When two spin-up electrons are placed in this potential, which one of the following, with x1 and x2 denoting the position of the two electrons, correctly represents the space part of the ground state wave function of the system?
\hfill{\brak{2014}}
\begin{enumerate}
\item $\frac{1}{\sqrt{2}}\sbrak{\psi_1\brak{x_1}\psi_2\brak{x_1} - \psi_1\brak{x_2}\psi_2\brak{x_2}}$
\item $\frac{1}{\sqrt{2}}\sbrak{\psi_1\brak{x_1}\psi_2\brak{x_2} + \psi_1\brak{x_2}\psi_2\brak{x_1}}$
\item $\frac{1}{\sqrt{2}}\sbrak{\psi_1\brak{x_1}\psi_2\brak{x_1} + \psi_1\brak{x_2}\psi_2\brak{x_2}}$
\item $\frac{1}{\sqrt{2}}\sbrak{\psi_1\brak{x_1}\psi_2\brak{x_2} - \psi_1\brak{x_2}\psi_2\brak{x_1}}$
\end{enumerate}

\item If the vector potential
\begin{align*}
\vec{A} = \alpha x\vec{x} + 2y\vec{y} - 3z\vec{z},
\end{align*}
satisfies the Colomb gauge, the value of the constant $\alpha$ is $\rule{2cm}{0.1pt}$
\hfill{\brak{2014}}

\item At a given temperature, $T$, the average energy per particle of a non-interacting gas of two-dimensional classical harmonic oscillators is $\rule{2cm}{0.1pt} k_{B}T\brak{k_B \text{ is the Boltzmann constant}}.$
\hfill{\brak{2014}}

\item Which of the following is a fermion?
\hfill{\brak{2014}}
\begin{enumerate}
\item $\alpha$ particle
\item $Be_{4}^{7}$ nucleus
\item Hydrogen atom
\item Deuteron
\end{enumerate}


\end{enumerate}
\end{document}

