%iffalse
\let\negmedspace\undefined
\let\negthickspace\undefined
\documentclass[journal,12pt,onecolumn]{IEEEtran}
\usepackage{cite}
\usepackage{amsmath,amssymb,amsfonts,amsthm}
\usepackage{algorithmic}
\usepackage{graphicx}
\usepackage{textcomp}
\usepackage{xcolor}
\usepackage{txfonts}
\usepackage{listings}
\usepackage{enumitem}
\usepackage{mathtools}
\usepackage{gensymb}
\usepackage{comment}
\usepackage[breaklinks=true]{hyperref}
\usepackage{tkz-euclide} 
\usepackage{listings}
\usepackage{gvv}
\usepackage{circuitikz}
%\def\inputGnumericTable{}                         
\usepackage[latin1]{inputenc}                                
\usepackage{color}                                            
\usepackage{array}                                            
\usepackage{longtable}                                       
\usepackage{calc}                                             
\usepackage{multirow}                                         
\usepackage{hhline}                                           
\usepackage{ifthen}                                           
\usepackage{lscape}
\usepackage{tabularx}
\usepackage{array}
\usepackage{float}
\usepackage{multicol}

\newtheorem{theorem}{Theorem}[section]
\newtheorem{problem}{Problem}
\newtheorem{proposition}{Proposition}[section]
\newtheorem{lemma}{Lemma}[section]
\newtheorem{corollary}[theorem]{Corollary}
\newtheorem{example}{Example}[section]
\newtheorem{definition}[problem]{Definition}
\newcommand{\BEQA}{\begin{eqnarray}}
\newcommand{\EEQA}{\end{eqnarray}}
\newcommand{\define}{\stackrel{\triangle}{=}}
\theoremstyle{remark}
\newtheorem{rem}{Remark}

\begin{document}
\bibliographystyle{IEEEtran}
\title{2015-AE-40-52}
\author{EE24BTECH11003 - Akshara Sarma Chennubhatla}
\maketitle
\begin{enumerate}[start=40]

\item The directional derivative of the field $u\brak{x,y,z} = x^2 - 3yz$ in the direction of the vector $\brak{\vec{i} + \vec{j} - 2\vec{k}}$ at point $\brak{2,-1,4}$ is $\rule{2cm}{0.1pt}$
\hfill{\brak{2015}}

\item The composition of an air-entrained concrete is given below:
\begin{align*}
\text{Water} &: 184 kg/m^3\\
\text{Ordinary Portland Cement}\brak{\text{OPC}} &: 368 kg/m^3\\
\text{Sand} &: 606 kg/m^3\\
\text{Coarse aggregate} &: 1155 kg/m^3\\
\end{align*}
Assume the specific gravity of OPC, sand and course aggregate to be $3.14$, $2.67$ and $2.74$, respectively. TYhe air content is $\rule{2cm}{0.1pt}$ litres/$m^3.$
\hfill{\brak{2015}}

\item A bracket plate connected to a column flange transmits a load of $100$ kN as shown in the following figure. The maximum force for which the bolts should be designed is $\rule{2cm}{0.1pt}$ kN.
\hfill{\brak{2015}}
\begin{center}
\begin{circuitikz}
\tikzstyle{every node}=[font=\large]
\draw  (4,15) rectangle (8,14.25);
\draw  (4,15) -- (8,15) -- (8.5,15.25) -- (4.5,15.25) -- cycle;
\draw [short] (8,14.25) -- (8.5,14.5);
\draw [short] (8.5,15.25) -- (8.5,14.5);
\draw [short] (4,14.5) -- (3.75,14.75);
\draw [short] (4,14.75) -- (3.75,15);
\draw [short] (4,14.5) -- (3.75,14.75);
\draw [short] (4,14.25) -- (3.75,14.5);
\draw [short] (4,15) -- (3.75,15.25);
\draw  (11.75,14.5) rectangle (15.75,15.25);
\draw  (11.75,15.25) -- (15.75,15.25) -- (16.25,15.5) -- (12.25,15.5) -- cycle;
\draw [short] (15.75,14.5) -- (16.25,14.75);
\draw [short] (16.25,15.5) -- (16.25,14.75);
\draw [->, >=Stealth] (8.25,14.75) -- (9.75,14.75);
\draw [->, >=Stealth] (17.5,15) -- (16,15);
\draw [short] (11.75,14.5) -- (11.5,14.75);
\draw [short] (11.75,14.75) -- (11.5,15);
\draw [short] (11.75,15) -- (11.5,15.25);
\draw [short] (11.75,15.25) -- (11.5,15.5);
\draw [short] (4,12) -- (8.5,12);
\draw [short] (4,11.25) -- (8.5,11.25);
\draw [short] (8.5,12) .. controls (8.5,12) and (8.25,11.5) .. (8.5,11.25);
\draw [short] (4,12) .. controls (3.75,11.75) and (3.75,11.25) .. (4,11.25);
\draw  (12,12) rectangle (16,11.25);
\draw  (12,12) -- (16,12) -- (16.5,12.25) -- (12.5,12.25) -- cycle;
\draw [short] (16,11.25) -- (16.5,11.5);
\draw [short] (16.5,12.25) -- (16.5,11.5);
\draw [short] (12,11.25) -- (11.75,11.5);
\draw [short] (12,11.5) -- (11.75,11.75);
\draw [short] (12,11.75) -- (11.75,12);
\draw [short] (12,12) -- (11.75,12.25);
\draw [->, >=Stealth] (16.5,11) .. controls (17.25,11.5) and (17,12.5) .. (16.25,12.5) ;
\draw [->, >=Stealth] (8.25,11) .. controls (9.5,11) and (8.25,13.25) .. (8,11.75) ;
\draw [short] (8.5,12) .. controls (8.75,11.75) and (8.75,11.25) .. (8.5,11.25);
\draw [short] (4,11.75) -- (3.75,12);
\draw [short] (4,11.5) -- (3.75,11.75);
\draw [short] (4,11.25) -- (3.75,11.5);
\node [font=\Large] at (6.25,15.75) {Tensile load};
\node [font=\large] at (9.25,14) {80 kN};
\node [font=\Large] at (14.25,16) {Compressive load};
\node [font=\large] at (17,14.25) {80 kN};
\node [font=\large] at (9.75,11.75) {64$\pi$ Nm};
\node [font=\large] at (17.75,12.5) {320 Nm};
\node [font=\Large] at (14.25,12.75) {Bending load};
\node [font=\Large] at (6.25,12.5) {Torsional load};
\end{circuitikz}

\end{center}

\item Consider the singly reinforced beam section given below $\brak{\text{left figure}}$. The stress block parameters for the cross-section from IS:$456-2000$ are also given below $\brak{\text{right figure}}$. The moment of resistance for the given section by the limit state method is $\rule{2cm}{0.1pt}$ kN-m.
\hfill{\brak{2015}}
\begin{center}
\begin{circuitikz}
\tikzstyle{every node}=[font=\large]
\draw [short] (11,18) -- (11,8.25);
\draw [->, >=Stealth] (9.25,10.25) -- (20.75,10.25);
\draw [->, >=Stealth] (9.25,14.75) -- (20.75,14.75);
\draw [line width=1.5pt, short] (11,10.25) -- (13,10.25);
\draw [line width=1.5pt, short] (13,10.25) -- (15,12.75);
\draw [line width=1.5pt, short] (15,12.75) -- (15,11.5);
\draw [line width=1.5pt, short] (15,11.5) -- (20.25,11.5);
\draw [line width=1.5pt, short] (11,17) -- (15,17);
\draw [line width=1.5pt, short] (15,17) -- (17,14.75);
\draw [line width=1.5pt, short] (17,14.75) -- (20.25,14.75);
\draw [dashed] (13,8) -- (13,18.75);
\draw [dashed] (15,18.75) -- (15,8);
\draw [dashed] (17,18.75) -- (17,8);
\draw [->, >=Stealth] (11.5,14.75) -- (11.5,17);
\draw [->, >=Stealth] (10.75,15.25) -- (10.75,16.5);
\draw [->, >=Stealth] (10.75,10.75) -- (10.75,12.5);
\draw [->, >=Stealth] (15.5,11.5) -- (15.5,12.75);
\draw [->, >=Stealth] (17.5,10.25) -- (17.5,11.5);
\draw [short] (15.25,12.75) -- (15.75,12.75);
\draw [<->, >=Stealth] (13,8.5) -- (15,8.5);
\draw [<->, >=Stealth] (15,8.5) -- (17,8.5);
\node [font=\large] at (10.25,11.25) {$i_s$};
\node [font=\large] at (10.25,15.75) {$v_s$};
\node [font=\large] at (12.25,15.75) {600 V};
\node [font=\large] at (16,12.25) {50 A};
\node [font=\large] at (18.25,10.75) {100 A};
\node [font=\large] at (14,7.75) {T1 = 1\mus};
\node [font=\large] at (16,7.75) {T2 = 1\mus};
\node [font=\large] at (21,14.75) {t};
\node [font=\large] at (21,10.25) {t};
\end{circuitikz}

\end{center}

\item For formation of collapse mechanism in the following figure, the minimum value of $P_u$ is $xM_p/L$. $M_p$ and $3M_p$ denote the plastic moment capacities of beam section as shown in this figure. The value of $c$ is $\rule{2cm}{0.1pt}.$
\hfill{\brak{2015}}
\begin{center}
\begin{center}
\begin{circuitikz}
\tikzstyle{every node}=[font=\LARGE]
\draw [short] (12.25,15.75) .. controls (12,6.5) and (19,9.25) .. (19.5,15);
\draw [short] (12.75,15.75) .. controls (13,6.25) and (19,11.25) .. (19,15);
\draw  (12.5,15.75) ellipse (0.3cm and 0.2cm);
\draw  (19.25,15) ellipse (0.3cm and 0.2cm);
\draw [line width=1.3pt, short] (19.55,14.9) -- (9.5,10.75);
\draw [line width=1.3pt, short] (12.25,15.75) -- (9,11.25);
\draw [ line width=2pt , rotate around={-288:(9.25,11)}] (9.25,11) ellipse (0.25cm and 0.5cm);
\draw [line width=1pt, dashed] (12.5,15.75) -- (19.25,15);
\draw [line width=1.2pt, dashed] (15.5,16.25) -- (15,6.75);
\draw [dashed] (8.5,10.8) -- (15.5,15.5);
\draw [short] (13.25,10.75) .. controls (14.25,9) and (14.25,9.25) .. (14.25,7);
\draw [short] (17,10.3) .. controls (15.75,9) and (15.75,9) .. (15.5,7);
\draw [short] (14.25,7) .. controls (15,6.5) and (15,6.75) .. (15.5,7);
\draw [->, >=Stealth] (15.25,15.5) -- (9.25,11.5);
\draw [->, >=Stealth] (9.5,11.25) -- (11,12.25);
\node [font=\large] at (11,11.75) {$F_x$};
\node [font=\Large] at (11.25,13.25) {x};
\draw [->, >=Stealth] (10.75,10.25) -- (9.25,11);
\node [font=\large] at (11.25,10) {Stone of mass 0.1 kg};
\draw [short] (9.25,11) -- (9.25,12);
\end{circuitikz}
\end{center}

\end{center}

\item A tapered circular rod of diameter varying from $20$ mm to $10$ mm is connected to another uniform circular rod of diameter $10$ mm as shown in the figure. Both bars are made of same material with the modulus of elasticity, $E = 2$ x $10^5$ MPa. When subjected to a load $P = 30\pi$ kN, the deflection at point $A$ is $\rule{2cm}{0.1pt}$ mm.
\hfill{\brak{2015}}
\begin{center}
\begin{circuitikz}
\tikzstyle{every node}=[font=\large]
\draw [short] (9.5,15.5) -- (13.75,15.5);
\draw [short] (9.75,15.5) -- (10.25,16);
\draw [short] (10.25,15.5) -- (10.75,16);
\draw [short] (10.75,15.5) -- (11.25,16);
\draw [short] (11.25,15.5) -- (11.75,16);
\draw [short] (11.75,15.5) -- (12.25,16);
\draw [short] (12.25,15.5) -- (12.75,16);
\draw [short] (12.75,15.5) -- (13.25,16);
\draw [short] (13.25,15.5) -- (13.75,16);
\draw [short] (11.75,15.5) -- (13.75,15.5);
\draw [short] (10,15.5) -- (10.75,12);
\draw [short] (13,15.5) -- (12.25,12);
\draw [short] (10.75,12) -- (12.25,12);
\draw [short] (10.75,12) -- (10.75,9.5);
\draw [short] (12.25,12) -- (12.25,9.5);
\draw [short] (10.75,9.5) -- (12.25,9.5);
\draw [<->, >=Stealth] (8.75,15.5) -- (8.75,11.75);
\draw [<->, >=Stealth] (8.75,11.75) -- (8.75,9.5);
\draw [line width=1.6pt, ->, >=Stealth] (11.5,9.5) -- (11.5,8.25);
\draw [<->, >=Stealth] (10,16.75) -- (13,16.75);
\draw [<->, >=Stealth] (10.75,11.5) -- (12.25,11.5);
\draw [short] (8.5,11.75) -- (9,11.75);
\draw [short] (8.5,15.5) -- (9,15.5);
\draw [short] (8.5,9.5) -- (9,9.5);
\draw [short] (10,17) -- (10,16.25);
\draw [short] (13,17) -- (13,16.25);
\node [font=\large] at (11.5,17.25) {$d_1 = 20 mm$};
\node [font=\large] at (9.25,13.5) {2 m};
\node [font=\large] at (9.5,10.5) {1.5 m};
\node [font=\large] at (13.5,11.5) {$d_2 = 10 mm$};
\node [font=\large] at (12.5,9.5) {A};
\node [font=\large] at (11.75,7.75) {P = 30$\pi$ kN};
\end{circuitikz}

\end{center}

\item Two beams are connected by a linear spring as shown in the following figure. For a load $P$ as shown in the figure, the percentage of the applied load $P$ carried by the spring is $\rule{2cm}{0.1pt}.$
\hfill{\brak{2015}}
\begin{center}
\begin{circuitikz}
\tikzstyle{every node}=[font=\normalsize]
\draw [ color={rgb,255:red,242; green,133; blue,0} ] (5,14.75) rectangle (17.5,14.5);
\draw [ color={rgb,255:red,242; green,133; blue,0} ] (6,14.5) rectangle (6.5,14.25);
\draw [ color={rgb,255:red,242; green,133; blue,0} ] (6.25,14.5) rectangle (6.75,14.25);
\draw [ color={rgb,255:red,242; green,133; blue,0} ] (9.75,14.5) rectangle (10.25,14.25);
\draw [ color={rgb,255:red,242; green,133; blue,0} ] (10,14.5) rectangle (10.5,14.25);
\draw [ color={rgb,255:red,0; green,84; blue,194}, short] (5,14.75) -- (5,16);
\draw [ color={rgb,255:red,0; green,84; blue,194}, short] (5,16) -- (5.25,16);
\draw [ color={rgb,255:red,0; green,84; blue,194}, short] (5.25,16) -- (5.5,15);
\draw [ color={rgb,255:red,0; green,84; blue,194}, short] (5.5,15) -- (11.25,15);
\draw [ color={rgb,255:red,0; green,84; blue,194}, short] (11.25,15) -- (11.5,16);
\draw [ color={rgb,255:red,0; green,84; blue,194}, short] (11.5,16) -- (11.75,16);
\draw [ color={rgb,255:red,0; green,84; blue,194}, short] (11.75,16) -- (11.75,14.75);
\draw [ color={rgb,255:red,242; green,133; blue,0} ] (6,15) rectangle (6.75,15.25);
\draw [ color={rgb,255:red,242; green,133; blue,0} ] (6.5,15) rectangle (6.25,15.5);
\draw [ color={rgb,255:red,242; green,133; blue,0} ] (9.75,15) rectangle (10.5,15.25);
\draw [ color={rgb,255:red,242; green,133; blue,0} ] (10,15) rectangle (10.25,15.5);
\draw [ color={rgb,255:red,242; green,133; blue,0} ] (5,12.25) rectangle (17.5,6);
\draw [ color={rgb,255:red,0; green,84; blue,194}, short] (5,12.25) -- (5,13);
\draw [ color={rgb,255:red,0; green,84; blue,194}, short] (11.75,12.25) -- (11.75,13);
\draw [ color={rgb,255:red,0; green,84; blue,194}, dashed] (6.25,15.75) -- (6.25,5.5);
\draw [ color={rgb,255:red,0; green,84; blue,194}, dashed] (10,15.75) -- (10,5.5);
\draw [ color={rgb,255:red,242; green,133; blue,0}, short] (6.25,11.25) -- (6,11.25);
\draw [ color={rgb,255:red,242; green,133; blue,0}, short] (6.25,11.25) -- (6.5,11.25);
\draw [ color={rgb,255:red,242; green,133; blue,0}, short] (6.5,11.25) -- (6.75,11);
\draw [ color={rgb,255:red,242; green,133; blue,0}, short] (6.75,11) -- (6.5,10.75);
\draw [ color={rgb,255:red,242; green,133; blue,0}, short] (6.5,10.75) -- (6,10.75);
\draw [ color={rgb,255:red,242; green,133; blue,0}, short] (6,10.75) -- (5.75,11);
\draw [ color={rgb,255:red,242; green,133; blue,0}, short] (5.75,11) -- (6,11.25);
\draw [ color={rgb,255:red,242; green,133; blue,0}, short] (9.75,11.25) -- (10.25,11.25);
\draw [ color={rgb,255:red,242; green,133; blue,0}, short] (10.25,11.25) -- (10.5,11);
\draw [ color={rgb,255:red,242; green,133; blue,0}, short] (10.5,11) -- (10.25,10.75);
\draw [ color={rgb,255:red,242; green,133; blue,0}, short] (10.25,10.75) -- (9.75,10.75);
\draw [ color={rgb,255:red,242; green,133; blue,0}, short] (9.75,10.75) -- (9.5,11);
\draw [ color={rgb,255:red,242; green,133; blue,0}, short] (9.5,11) -- (9.75,11.25);
\draw [ color={rgb,255:red,242; green,133; blue,0}, short] (6,7.5) -- (6.5,7.5);
\draw [ color={rgb,255:red,242; green,133; blue,0}, short] (6.5,7.5) -- (6.75,7.25);
\draw [ color={rgb,255:red,242; green,133; blue,0}, short] (6.75,7.25) -- (6.5,7);
\draw [ color={rgb,255:red,242; green,133; blue,0}, short] (6.5,7) -- (6,7);
\draw [ color={rgb,255:red,242; green,133; blue,0}, short] (6,7) -- (5.75,7.25);
\draw [ color={rgb,255:red,242; green,133; blue,0}, short] (5.75,7.25) -- (6,7.5);
\draw [ color={rgb,255:red,242; green,133; blue,0}, short] (9.75,7.5) -- (10.25,7.5);
\draw [ color={rgb,255:red,242; green,133; blue,0}, short] (10.25,7.5) -- (10.5,7.25);
\draw [ color={rgb,255:red,242; green,133; blue,0}, short] (10.5,7.25) -- (10.25,7);
\draw [ color={rgb,255:red,242; green,133; blue,0}, short] (10.25,7) -- (9.75,7);
\draw [ color={rgb,255:red,242; green,133; blue,0}, short] (9.75,7) -- (9.5,7.25);
\draw [ color={rgb,255:red,242; green,133; blue,0}, short] (9.5,7.25) -- (9.75,7.5);
\draw [ color={rgb,255:red,0; green,84; blue,194}, dashed] (5.5,11) -- (13.75,11);
\draw [ color={rgb,255:red,0; green,84; blue,194}, dashed] (5.5,7.25) -- (13.75,7.25);
\draw [ color={rgb,255:red,0; green,84; blue,194}, dashed] (6.75,9.25) -- (18,9.25);
\draw [ color={rgb,255:red,0; green,84; blue,194}, dashed] (8.25,5.75) -- (10.25,5.75);
\draw [ color={rgb,255:red,0; green,84; blue,194}, dashed] (9.25,15.25) -- (11.25,15.25);
\draw [short] (5,13) .. controls (5.5,13.25) and (5.75,13) .. (6.25,13);
\draw [short] (6.25,13) .. controls (7,13.25) and (7.25,13.25) .. (8,13.25);
\draw [short] (8,13.25) .. controls (8.75,13) and (9,13) .. (10,13);
\draw [short] (10,13) -- (11.75,13.25);
\draw [short] (5,5.5) .. controls (5.75,5.25) and (5.5,5.25) .. (6.25,5.25);
\draw [short] (6.25,5.25) .. controls (7,5) and (7,5.5) .. (8,5.5);
\draw [short] (8,5.5) -- (10,5.5);
\draw [ color={rgb,255:red,0; green,84; blue,194}, short] (10,5.5) -- (11.25,5.5);
\draw [ color={rgb,255:red,0; green,84; blue, 194}, short] (11.25,5.5) -- (11.25,6);
\draw [ color={rgb,255:red,0; green,84; blue, 194}, short] (5,5.5) -- (5,6);
\draw [ color={rgb,255:red,0; green,84; blue, 194}, short] (11.75,13.25) -- (11.75,13);
\draw [ color={rgb,255:red,0; green,84; blue, 194}, <->, >=Stealth] (5,4.75) -- (8,4.75);
\draw [ color={rgb,255:red,0; green,84; blue, 194}, <->, >=Stealth] (8.25,4.75) -- (11.25,4.75);
\draw [ color={rgb,255:red,0; green,84; blue, 194}, <->, >=Stealth] (11.25,4.75) -- (18.75,4.75);
\draw [ color={rgb,255:red,0; green,84; blue, 194}, <->, >=Stealth] (18,12.25) -- (18,6);
\node [font=\normalsize, color={rgb,255:red,0; green,84; blue, 194}] at (18.25,9.5) {200};
\node [font=\normalsize] at (11,10.25) {50};
\node [font=\normalsize] at (11,8.25) {50};
\node [font=\normalsize] at (5.75,11.5) {A};
\node [font=\normalsize] at (9.5,11.5) {B};
\node [font=\normalsize] at (8.25,9.5) {O};
\node [font=\normalsize] at (5.75,7.75) {C};
\node [font=\normalsize] at (9.5,7.75) {D};
\node [font=\normalsize] at (6.25,4.5) {100};
\node [font=\normalsize] at (9.75,4.5) {100};
\node [font=\normalsize] at (14.5,4.25) {300};
\node [font=\normalsize] at (15.25,13) {F=10kN};
\node [font=\normalsize] at (8.5,16.75) {200};
\node [font=\normalsize] at (12.75,15.75) {Steel channel};
\node [font=\normalsize] at (16.25,16.75) {Not to scale};
\node [font=\normalsize] at (16.25,16.25) {Dimensions in mm};
\draw [ color={rgb,255:red,0; green,84; blue,194}, <->, >=Stealth] (5.25,16.5) -- (11.5,16.5);
\draw [ color={rgb,255:red,0; green,84; blue,194}, ->, >=Stealth] (16.25,14) -- (16.25,12.25);
\draw [ color={rgb,255:red,0; green,84; blue,194}, <->, >=Stealth] (10.75,11) -- (10.75,9.25);
\draw [ color={rgb,255:red,0; green,84; blue,194}, <->, >=Stealth] (10.75,9.25) -- (10.75,7.25);
\end{circuitikz}

\end{center}

\item For the $2D$ truss with the applied loads shown below, the strain energy in the member $XY$ is $\rule{2cm}{0.1pt}$ kN-m. For the member $XY$, assume $AE = 30$ kN, where $A$ is cross-section area and $E$ is the modulus of elasticity.
\hfill{\brak{2015}}
\begin{center}
\begin{circuitikz}
\tikzstyle{every node}=[font=\Large]
\draw [short] (9.5,12.25) -- (13.25,12.25);
\draw [short] (9.5,10) -- (15.25,10);
\draw [short] (13.25,12.25) -- (13.25,16.25);
\draw [short] (15.25,16.25) -- (15.25,10);
\draw [short] (13.25,16.25) -- (15.25,16.25);
\draw [short] (13.25,19.5) -- (15.25,17.5);
\draw [->, >=Stealth] (7.75,12.25) -- (9.5,12.25);
\draw [->, >=Stealth] (7.75,12) -- (9.5,12);
\draw [->, >=Stealth] (7.75,11.75) -- (9.5,11.75);
\draw [->, >=Stealth] (7.75,11.5) -- (9.5,11.5);
\draw [->, >=Stealth] (7.75,11.25) -- (9.5,11.25);
\draw [->, >=Stealth] (7.75,11) -- (9.5,11);
\draw [->, >=Stealth] (7.75,10.75) -- (9.5,10.75);
\draw [->, >=Stealth] (7.75,10.5) -- (9.5,10.5);
\draw [->, >=Stealth] (7.75,10.25) -- (9.5,10.25);
\draw [->, >=Stealth] (7.75,10) -- (9.5,10);
\draw [->, >=Stealth] (13.25,16.25) -- (13.25,19.5);
\draw [->, >=Stealth] (13.75,16.25) -- (13.75,19);
\draw [->, >=Stealth] (15.25,16.25) -- (15.25,17.5);
\draw [->, >=Stealth] (14.75,16.25) -- (14.75,18);
\draw [->, >=Stealth] (14.25,16.25) -- (14.25,18.5);
\draw [short] (7.75,12.25) -- (7.75,10);
\draw [short] (9.5,12.25) -- (9.5,10);
\draw [->, >=Stealth] (15,16.25) -- (15,17.75);
\draw [->, >=Stealth] (14.5,16.25) -- (14.5,18.25);
\draw [->, >=Stealth] (14,16.25) -- (14,18.75);
\draw [->, >=Stealth] (13.5,16.25) -- (13.5,19.25);
\draw [<->, >=Stealth] (13.25,14.25) -- (15.25,14.25);
\draw [<->, >=Stealth] (10.75,12.25) -- (10.75,10);
\node [font=\Large] at (6.5,11.25) {5 m/s};
\node [font=\Large] at (12,11.25) {50 mm};
\node [font=\Large] at (14.25,14.75) {60 mm};
\node [font=\Large] at (12.5,17.75) {$V_{max}$};
\node [font=\Large] at (16,16.75) {$V_{min}$};
\end{circuitikz}

\end{center}

\item An earth embankment is to be constructed with compacted codesionless soil. The volume of the embankment is $5000m^3$ and the target dry unit weight is $16.2$ kN/$m^3$. Three nearby sites, see the figure below, have been identified from where the required soil can be transported to the construction site. The void ratios $\brak{e}$ of different sites are shown in the figure. Assume the specific gravity of soil to be $2.7$ for all three sites. If the cost of transportation per km is twice the cost of excavation per $m^3$ of borrow pits, which site would you choose as the most economic solution?$\brak{\text{Use unit weight of water} = 10kN/m^3}$
\hfill{\brak{2015}}
\begin{enumerate}
\item Site $X$
\item Site $Y$
\item Site $Z$
\item Any of the sites
\end{enumerate}
\begin{center}
\begin{circuitikz}
\tikzstyle{every node}=[font=\large]
\draw [short] (10.5,12.5) -- (11.75,13.5);
\draw [short] (11.75,13.5) -- (13.25,12.5);
\draw [short] (10.5,12.5) -- (11.25,11);
\draw [short] (13.25,12.5) -- (12.5,11);
\draw [short] (11.25,11) -- (12.5,11);
\draw [<->, >=Stealth] (11.75,13.5) -- (11.75,16.25);
\draw [<->, >=Stealth] (11.25,11) -- (9.75,9.75);
\draw [<->, >=Stealth] (12.5,11) -- (14.5,9.75);
\draw  (11.75,17.25) circle (1cm);
\draw  (9,9) circle (1cm);
\draw  (15.25,9) circle (1cm);
\node [font=\large] at (12.5,14.75) {140 km};
\node [font=\large] at (11.75,17.75) {Site X};
\node [font=\large] at (12,12.5) {Construction };
\node [font=\large] at (10,10.75) {100 km};
\node [font=\large] at (14,10.75) {80 km};
\node [font=\large] at (9,9.5) {Site Z};
\node [font=\large] at (15.25,9.5) {Site Y};
\node [font=\large] at (11.75,17) {e = 0.6};
\node [font=\large] at (9,8.75) {e = 0.64};
\node [font=\large] at (15.25,8.75) {e = 0.7};
\node [font=\large] at (12,11.75) {Site };
\end{circuitikz}

\end{center}

\item A water tank is to be constructed on th soil deposit shown in the figure below. A circular footing of diameter $3$ m and depth of embedment $1$ m has been designed to support the tank. The total vertical load to be taken by the footing is $1500$ kN. Assume the unit weight of water as $10$ kN/$m^3$ and the load dispersion pattern as $2V:1H$. The expected settlement of the tank due to primary consolidation of the clay layer is $\rule{2cm}{0.1pt}$ mm.
\hfill{\brak{2015}}
\begin{center}
\begin{circuitikz}
\tikzstyle{every node}=[font=\large]
\draw [line width=1.6pt, short] (7.5,15) -- (17.75,15);
\draw [line width=1.6pt, short] (7.5,13.75) -- (17.75,13.75);
\draw [line width=1.6pt, short] (7.5,11.75) -- (18,11.75);
\draw [line width=1.6pt, short] (7.5,9) -- (18,9);
\draw [<->, >=Stealth] (8,15) -- (8,13.75);
\draw [<->, >=Stealth] (8,13.75) -- (8,11.75);
\draw [<->, >=Stealth] (8,11.75) -- (8,9);
\draw [->, >=Stealth] (8,9) -- (8,7.25);
\node [font=\large] at (7.25,14.25) {2 m};
\node [font=\large] at (7.25,12.75) {6 m};
\node [font=\large] at (7.25,10.25) {10 m};
\node [font=\large] at (9.25,14.25) {Silty sand};
\node [font=\large] at (8.75,12.75) {Sand};
\node [font=\large] at (10,10.75) {Normally consolidated };
\node [font=\large] at (9.25,8) {Dense sand};
\node [font=\large] at (18,14.25) {$\text{Bulk unit weight} = 15 kN/m^3$};
\node [font=\large] at (18,12.75) {$\text{Saturated unit weight}= 18 kN/m^3$};
\node [font=\large] at (18,11) {$\text{Saturated unit weight} = 18 kN/m^3$};
\node [font=\large] at (9,10) {clay};
\node [font=\large] at (18,10.5) {$\text{Compression index} = 0.3$};
\node [font=\large] at (18,10) {$\text{Initial void ratio} = 0.7$};
\node [font=\large] at (18,9.5) {$\text{Coefficient of consolidation}  = 0.004 cm^2/s$};
\end{circuitikz}

\end{center}

\item A $20$ m thick clay layer is sandwiched between a silty sand layer and a gravelly sand layer. The layer experiences $30$ mm settlement in $2$ years.\\
Given:\\
$T_v = 
\begin{cases}
\frac{\pi}{4}\brak{\frac{U}{100}}^2 & for U \leq 60\% \\
1.781 - 0.933\log_{10}\brak{100-U} & U > 60\%\\
\end{cases}$\\
where $T_v$ is the time factor and $U$ is the degree of consolidation in $\%$.\\
If the coefficient of consolidation of the layer is $0.003$ $cm^2/s$, the deposit will experience a total of $50$ mm settlement in the next $\rule{2cm}{0.1pt}$ years.
\hfill{\brak{2015}}

\item A non-homogeneous soil deposit consiste of a silt later sandwiched between a fine-sand layer at top and a clat later below. Permeability of the silt layer is $10$ times the permeability of the clay layer and one-tenth of the permeability of the sand layer. Thickness of the silt layer is $2$ times the thickness of the sand layer and two-third of the thickness of the clay layer. The ratio of equivalent horizontal and equivalent vertical permeability of the deposit is $\rule{2cm}{0.1pt}.$
\hfill{\brak{2015}}

\item A square footing, $2$ m x $2$ m, is subjected to an inclined point load, $P$ as shown in the figure below. The water table is located below the base of the footing. Considering one-way eccentricity, the net safe load carrying capacity of the footing for a factor of safety of $3.0$ is $\rule{2cm}{0.1pt}$ kN.\\
The following factors may be used:\\
Bearing capacity factors: $N_q = 33.3, N_\gamma = 37.16;$ Shape factors: $F_{qs} = F_{\gamma s} = 1.314;$ Depth factors: $F_{qd} = F_{\gamma d} = 1.113;$ Inclination factors: $F_{qi} = 0.444, F_{\gamma i} = 0.02$
\hfill{\brak{2015}}
\begin{center}
\begin{circuitikz}
\tikzstyle{every node}=[font=\large]
\draw [short] (7.25,14.25) -- (7.25,12.75);
\draw [short] (7.25,12.75) -- (6,12.75);
\draw [short] (6,12.75) -- (6,11.5);
\draw [short] (6,11.5) -- (11.75,11.5);
\draw [short] (8.25,14.25) -- (8.25,12.75);
\draw [short] (8.25,12.75) -- (11.75,12.75);
\draw [short] (11.75,12.75) -- (11.75,11.5);
\draw [->, >=Stealth] (5.75,14) -- (7.75,11.5);
\draw [dashed] (7.75,14.5) -- (7.75,10.5);
\draw [short] (7.25,12) .. controls (7.25,12.25) and (7.5,12.25) .. (7.75,12.25);
\draw [short] (12.75,12.75) -- (18.5,12.75);
\draw [short] (12.75,11.5) -- (14.25,11.5);
\draw [short] (6,11.25) -- (6,9.5);
\draw [short] (11.75,11.25) -- (11.75,9.5);
\draw [<->, >=Stealth] (6,11) -- (7.75,11);
\draw [<->, >=Stealth] (6,10) -- (11.75,10);
\draw [<->, >=Stealth] (13.25,12.75) -- (13.25,11.5);
\draw [short] (14.5,12.5) -- (14.75,12.75);
\draw [short] (14.75,12.75) -- (15,12.5);
\draw [short] (14.75,12.5) -- (15,12.75);
\draw [short] (15,12.75) -- (15.25,12.5);
\node [font=\large] at (6.5,13.5) {P};
\node [font=\large] at (7.25,12.5) {$30^\circ$};
\node [font=\large] at (7,11.25) {0.85 m};
\node [font=\large] at (8.75,10.25) {2 m};
\node [font=\large] at (13.75,12) {1 m};
\node [font=\large] at (14,13.25) {GL};
\node [font=\large] at (14.5,11) {$Unit weight = 18 kN/m^3$};
\node [font=\large] at (13.5,10.5) {Cohesion = 0};
\node [font=\large] at (14,10) {$Friction angle = 35^\circ$};
\end{circuitikz}

\end{center}


\end{enumerate}
\end{document}
