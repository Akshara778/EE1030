%iffalse
\let\negmedspace\undefined
\let\negthickspace\undefined
\documentclass[journal,12pt,onecolumn]{IEEEtran}
\usepackage{cite}
\usepackage{amsmath,amssymb,amsfonts,amsthm}
\usepackage{algorithmic}
\usepackage{graphicx}
\usepackage{textcomp}
\usepackage{xcolor}
\usepackage{txfonts}
\usepackage{listings}
\usepackage{enumitem}
\usepackage{mathtools}
\usepackage{gensymb}
\usepackage{comment}
\usepackage[breaklinks=true]{hyperref}
\usepackage{tkz-euclide} 
\usepackage{listings}
\usepackage{gvv}
\usepackage{circuitikz}
%\def\inputGnumericTable{}                         
\usepackage[latin1]{inputenc}                                
\usepackage{color}                                            
\usepackage{array}                                            
\usepackage{longtable}                                       
\usepackage{calc}                                             
\usepackage{multirow}                                         
\usepackage{hhline}                                           
\usepackage{ifthen}                                           
\usepackage{lscape}
\usepackage{tabularx}
\usepackage{array}
\usepackage{float}
\usepackage{multicol}

\newtheorem{theorem}{Theorem}[section]
\newtheorem{problem}{Problem}
\newtheorem{proposition}{Proposition}[section]
\newtheorem{lemma}{Lemma}[section]
\newtheorem{corollary}[theorem]{Corollary}
\newtheorem{example}{Example}[section]
\newtheorem{definition}[problem]{Definition}
\newcommand{\BEQA}{\begin{eqnarray}}
\newcommand{\EEQA}{\end{eqnarray}}
\newcommand{\define}{\stackrel{\triangle}{=}}
\theoremstyle{remark}
\newtheorem{rem}{Remark}

\begin{document}
\bibliographystyle{IEEEtran}
\title{2015-AE-40-52}
\author{EE24BTECH11003 - Akshara Sarma Chennubhatla}
\maketitle
\begin{enumerate}[start=40]

\item A cube made of linear elastic isotropic material is subjected to a uniform hydrostatic pressure of $100$ N/$mm^2$. Under this load, the volume of the cube shrinks by $0.05\%$. The Young's modulus of the material, $E=300$ GPa. The Poisson's ratio of the material is $\rule{2cm}{0.1pt}$.
\hfill{\brak{2015}}

\item A massless cantilever beam $PQ$ has a solid square cross section $\brak{10\text{ mm x }10\text{ mm}}$. This beam is subjected to a load $W$ through a rigid massless link at the point $Q$, as shown below $\brak{\text{figure not to scale}}$. If the Young's modulus of the material $E=200$ GPa, the deflection $\brak{\text{in mm}}$ at point $Q$ is $\rule{2cm}{0.1pt}$.
\hfill{\brak{2015}}
\begin{center}
\begin{circuitikz}
\tikzstyle{every node}=[font=\large]
\draw  (4,15) rectangle (8,14.25);
\draw  (4,15) -- (8,15) -- (8.5,15.25) -- (4.5,15.25) -- cycle;
\draw [short] (8,14.25) -- (8.5,14.5);
\draw [short] (8.5,15.25) -- (8.5,14.5);
\draw [short] (4,14.5) -- (3.75,14.75);
\draw [short] (4,14.75) -- (3.75,15);
\draw [short] (4,14.5) -- (3.75,14.75);
\draw [short] (4,14.25) -- (3.75,14.5);
\draw [short] (4,15) -- (3.75,15.25);
\draw  (11.75,14.5) rectangle (15.75,15.25);
\draw  (11.75,15.25) -- (15.75,15.25) -- (16.25,15.5) -- (12.25,15.5) -- cycle;
\draw [short] (15.75,14.5) -- (16.25,14.75);
\draw [short] (16.25,15.5) -- (16.25,14.75);
\draw [->, >=Stealth] (8.25,14.75) -- (9.75,14.75);
\draw [->, >=Stealth] (17.5,15) -- (16,15);
\draw [short] (11.75,14.5) -- (11.5,14.75);
\draw [short] (11.75,14.75) -- (11.5,15);
\draw [short] (11.75,15) -- (11.5,15.25);
\draw [short] (11.75,15.25) -- (11.5,15.5);
\draw [short] (4,12) -- (8.5,12);
\draw [short] (4,11.25) -- (8.5,11.25);
\draw [short] (8.5,12) .. controls (8.5,12) and (8.25,11.5) .. (8.5,11.25);
\draw [short] (4,12) .. controls (3.75,11.75) and (3.75,11.25) .. (4,11.25);
\draw  (12,12) rectangle (16,11.25);
\draw  (12,12) -- (16,12) -- (16.5,12.25) -- (12.5,12.25) -- cycle;
\draw [short] (16,11.25) -- (16.5,11.5);
\draw [short] (16.5,12.25) -- (16.5,11.5);
\draw [short] (12,11.25) -- (11.75,11.5);
\draw [short] (12,11.5) -- (11.75,11.75);
\draw [short] (12,11.75) -- (11.75,12);
\draw [short] (12,12) -- (11.75,12.25);
\draw [->, >=Stealth] (16.5,11) .. controls (17.25,11.5) and (17,12.5) .. (16.25,12.5) ;
\draw [->, >=Stealth] (8.25,11) .. controls (9.5,11) and (8.25,13.25) .. (8,11.75) ;
\draw [short] (8.5,12) .. controls (8.75,11.75) and (8.75,11.25) .. (8.5,11.25);
\draw [short] (4,11.75) -- (3.75,12);
\draw [short] (4,11.5) -- (3.75,11.75);
\draw [short] (4,11.25) -- (3.75,11.5);
\node [font=\Large] at (6.25,15.75) {Tensile load};
\node [font=\large] at (9.25,14) {80 kN};
\node [font=\Large] at (14.25,16) {Compressive load};
\node [font=\large] at (17,14.25) {80 kN};
\node [font=\large] at (9.75,11.75) {64$\pi$ Nm};
\node [font=\large] at (17.75,12.5) {320 Nm};
\node [font=\Large] at (14.25,12.75) {Bending load};
\node [font=\Large] at (6.25,12.5) {Torsional load};
\end{circuitikz}

\end{center}

\item An aircraft, with a wing loading $\frac{W}{S} = 500$ N/$m^2$, is gliding at $\brak{\frac{L}{D}}_{max} = 10$ and $C_L = 0.69$. Considering the free stream density $\rho_{\infty} = 0.9$ kg/$m^3$, the equilibrium glide speed $\brak{\text{in m/s}}$ is $\rule{2cm}{0.1pt}.$
\hfill{\brak{2015}}

\item For a thin flat plate at $2$ degress angle of attack, the pitching moment coefficient about the training edge is $\rule{2cm}{0.1pt}.$
\hfill{\brak{2015}}

\item A satellite is to be transferred from its geostationary orbit to a circular polar orbit of the same radius through a single impulse out-of-plane maneuver. The magnitude of the change in velocity required is $\rule{2cm}{0.1pt}$ times the magnitude of the escape velocity.
\hfill{\brak{2015}}

\item A planetary probe is launched at a speed of $200$ km/s and at a distance of $71,400$ km from the mass center of its nearest planet of mass $1.9$ x $10^{28}$ kg. The universal gravitational constant $G = 6.67$ x $10^{-11} \frac{m^3}{kg s^2}$. The ensuing path of the probe would be
\hfill{\brak{2015}}
\begin{enumerate}
\item ellipse
\item hyperbolic
\item parabolic
\item circular
\end{enumerate}

\item The velocity of an incompressible laminar boundary layer over a flat plate developing under constant pressure is given by $\frac{u\brak{y}}{U_{\infty}} = \frac{3y}{2\delta} - \frac{1}{2}\brak{\frac{y}{\delta}}^3$. The freestream velocity $U_\infty = 10$ m/s and the dynamic viscosity of the fluid $\mu = 1.8$ x $10^{-5}\frac{kg}{ms}$. At a streamwise station where the boundary layer thickness $\delta = 5$ mm, the wall shear stress is $\rule{2cm}{0.1pt}$ x $10^{-3}$ Pa.
\hfill{\brak{2015}}

\item the Pitot tube of an aircraft registers a pressure $p_0 = 54051 N/m^2$. The static pressure, density and the ratio of specific heats of the freestream are $p_\infty = 45565 N/m^2$, $\rho_\infty = 0.6417 kb/m^3$ and $\gamma = 1.4$, respectively. The indicated airspeed $\brak{\text{in m/s}}$ is
\hfill{\brak{2015}}
\begin{enumerate}
\item $157.6$
\item $162.6$
\item $172.0$
\item $182.3$
\end{enumerate}

\item Consider a NACA 0012 aerfoil of chord $c$ in a freestream with velocity $V_\infty$ at a non-zero positive angle of attack $\alpha$. The average time-of-flight for a particle to move from the leading edge to the trailing edge on the suction and pressure sides are $t_1$ and $t_2$, respectively. Thin aerfoil theory yields the velocity perturbation to the freestream as $V_\infty\frac{\brak{1 + \cos\theta}\alpha}{\sin\theta}$ on the suction side and as $-V_\infty\frac{\brak{1 + \cos\theta}\alpha}{\sin\theta}$ on the pressure side, where $\theta$ corresponds to the chordwise position $x = \frac{c}{2}\brak{1-\cos\theta}$. Then $t_2 - t_1$ is
\hfill{\brak{2015}}
\begin{enumerate}
\item $-\frac{8\pi\alpha c}{V_\infty \brak{4 - \pi^{2}\alpha^{2}}}$
\item $0$ 
\item $\frac{4\pi\alpha c}{V_\infty \brak{4 - \pi^{2}\alpha^{2}}}$
\item $\frac{8\pi\alpha c}{V_\infty \brak{4 - \pi^{2}\alpha^{2}}}$
\end{enumerate}

\item Air enters an aircraft engine at a velocity of $180$ m/s with a flow rate of $94$ kg/s. The engine combustor requires $9.2$ kg/s of air to burn $1$ kg/s of fuel. The velocity of gas exiting from the engine is $640$ m/s. The momentum thrust $\brak{\text{in N}}$ developed by the engine is
\hfill{\brak{2015}}
\begin{enumerate}
\item $43241$
\item $45594$
\item $47940$
\item $49779$
\end{enumerate}

\item A solid rocket monitor is designed with a cylindrical end-burning propellent grain of length $1$ m and diameter $32$ cm. The density of the propellent grain is $1750 kg/m^3$. The specific impulse of the motor is $190$ s and the acceleration due to the gravity is $9.8 m/s^2$. Tf the propellent burns for a period of $150$ s, then the thrust $\brak{\text{in N}}$ produced by the rocket motor is $\rule{2cm}{0.1pt}$
\hfill{\brak{2015}}

\item A liquid propellent rocket has the following component masses:
\begin{align*}
\text{Mass of payload} &= 180 kg\\
\text{Mass of fuel} &= 470 kg\\
\text{Mass of oxidizer} &= 1170 kg\\
\text{Mass of structures} &= 150 kg\\
\text{Mass of guidance systems} &= 20 kg\\
\end{align*}
The effective exhaust velocity is $3136$ m/s. The velocity increment $\brak{\text{in km/s}}$ of the rocket at burnout, while operating in outer space, is $\rule{2cm}{0.1pt}.$
\hfill{\brak{2015}}

\item If all the eigenvalues of a matrix are real and ewual, then
\hfill{\brak{2015}}
\begin{enumerate}
\item the matrix is diagonalizable
\item its eigenvectors are not necessarily linearly independent
\item its eigenvectors are linearly independent
\item its determinant is necessarily zero
\end{enumerate}


\end{enumerate}
\end{document}

