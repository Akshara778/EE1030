%iffalse
\let\negmedspace\undefined
\let\negthickspace\undefined
\documentclass[journal,12pt,onecolumn]{IEEEtran}
\usepackage{cite}
\usepackage{amsmath,amssymb,amsfonts,amsthm}
\usepackage{algorithmic}
\usepackage{graphicx}
\usepackage{textcomp}
\usepackage{xcolor}
\usepackage{txfonts}
\usepackage{listings}
\usepackage{enumitem}
\usepackage{mathtools}
\usepackage{gensymb}
\usepackage{comment}
\usepackage[breaklinks=true]{hyperref}
\usepackage{tkz-euclide} 
\usepackage{listings}
\usepackage{gvv}                                        
%\def\inputGnumericTable{}                         
\usepackage[latin1]{inputenc}                                
\usepackage{color}                                            
\usepackage{array}                                            
\usepackage{longtable}                                       
\usepackage{calc}                                             
\usepackage{multirow}                                         
\usepackage{hhline}                                           
\usepackage{ifthen}                                           
\usepackage{lscape}
\usepackage{tabularx}
\usepackage{array}
\usepackage{float}
\usepackage{multicol}

\newtheorem{theorem}{Theorem}[section]
\newtheorem{problem}{Problem}
\newtheorem{proposition}{Proposition}[section]
\newtheorem{lemma}{Lemma}[section]
\newtheorem{corollary}[theorem]{Corollary}
\newtheorem{example}{Example}[section]
\newtheorem{definition}[problem]{Definition}
\newcommand{\BEQA}{\begin{eqnarray}}
\newcommand{\EEQA}{\end{eqnarray}}
\newcommand{\define}{\stackrel{\triangle}{=}}
\theoremstyle{remark}
\newtheorem{rem}{Remark}

\begin{document}
\bibliographystyle{IEEEtran}
\title{2011-MA-1-13}
\author{EE24BTECH11003 - Akshara Sarma Chennubhatla% <-this % stops a space
}
\maketitle
\begin{enumerate}

\item The distinct eigenvalues of the matrix
$\myvec{1 & 1 & 0\\1 & 1 & 0\\0 & 0 & 0}$ are
\begin{enumerate}
\item $0$ and $1$
\item $1$ and $-1$
\item $1$ and $2$
\item $0$ and $2$
\end{enumerate}

\item The minimal polynomial of the matrix
$\myvec{3 & 3 & 0\\3 & 3 & 0\\0 & 0 & 6}$ is
\begin{enumerate}
\item $x\brak{x-1}\brak{x-6}$
\item $x\brak{x-3}$
\item $\brak{x-3}\brak{x-6}$
\item $x\brak{x-6}$
\end{enumerate}

\item Which of the following is the imaginary part of a possible value of $\ln \brak{\sqrt{i}}$?
\begin{enumerate}
\item $\pi$
\item $\frac{\pi}{2}$
\item $\frac{\pi}{4}$
\item $\frac{\pi}{8}$
\end{enumerate}

\item Let $f\colon \mathbb{C} \to \mathbb{C}$ be analytic except for a simple pole at $z=0$ and let $g\colon \mathbb{C} \to \mathbb{C}$ be analytic. Then, the value of $\frac{Res_{z=0}\cbrak{f\brak{z}g\brak{z}}}{Res_{z=0}\cbrak{f\brak{z}}}$ is
\begin{enumerate}
\item $g\brak{0}$
\item $g'\brak{0}$
\item $lim_{z \to 0} zf\brak{z}$
\item $lim_{z \to 0} zf\brak{z}g\brak{z}$
\end{enumerate}

\item Let $I = \oint_{C}\brak{2x^2 + y^2}dx + e^{y}dy$, where $C$ is the boundary $\brak{\text{oriented anticlockwise}}$ of the region in the first quadrant bounded by $y=0,x^2 + y^2 = 1 \text{ and } x=0$. The value of $I$ is
\begin{enumerate}
\item $-1$
\item $-\frac{2}{3}$
\item $\frac{2}{3}$
\item $1$
\end{enumerate}

\item The series $\sum_{1}^{\infty} x^{ln\brak{m}}$, $x>0$, is convergent on the interval
\begin{enumerate}
\item $\brak{0,\frac{1}{e}}$
\item $\brak{\frac{1}{e},e}$
\item $\brak{0,e}$
\item $\brak{1,e}$
\end{enumerate}

\item While solving the equation $x^2 - 3x + 1=0$ using the Newton-Raphson method with the initial guess of a root as $1$, the value of the root after one iteration is
\begin{enumerate}
\item $1.5$
\item $1$
\item $0.5$
\item $0$
\end{enumerate}

\item Consider the system of equations
$\myvec{5 & 2 & 1\\-2 & 5 & 2\\-1 & 2 & 8}\myvec{x_1\\x_2\\x_3}=\myvec{13\\-22\\14}$.
With the initial guess of the solution $\sbrak{x_{1}^{\brak{0}},x_{2}^{\brak{0}},x_{3}^{\brak{0}}}^{T} = \sbrak{1,1,1}^{T}$, the approximate value of the solution $\sbrak{x_{1}^{\brak{0}},x_{2}^{\brak{0}},x_{3}^{\brak{0}}}^{T}$ after one iteration by the Gauss-Seidel method is
\begin{enumerate}
\item $\sbrak{2,-4.4,1.625}^{T}$
\item $\sbrak{2,-4,-3}^{T}$
\item $\sbrak{2,4.4,1.625}^{T}$
\item $\sbrak{2,-4,3}^{T}$
\end{enumerate}

\item Let $y$ be the solution of the initial value problem
\begin{align*}
\frac{dy}{dx} = \brak{y^{2} + x}; y\brak{0}=1
\end{align*}
Using Taylor series method of order $2$ with the step size $h=0.1$, the approximate value of $y\brak{0.1}$ is
\begin{enumerate}
\item $1.315$
\item $1.415$
\item $1.115$
\item $1.215$
\end{enumerate}

\item The partial differential equation
\begin{align*}
x^{2} \frac{\partial^{2}{z}}{\partial{x^2}} - \brak{y^2 - 1}x\frac{\partial^{2}{z}}{\partial{x}\partial{y}} + y\brak{y-1}^{2}\frac{\partial^{2}{z}}{\partial{y^2}} + x\frac{\partial{z}}{\partial{x}} + y\frac{\partial{z}}{\partial{y}}=0 
\end{align*}
\text{is hyperbolic in a region in the} XY- \text{plane if}
\begin{enumerate}
\item $x\neq0 \text{ and } y=1$
\item $x=0 \text{ and } y\neq1$
\item $x\neq0 \text{ and } y\neq1$
\item $x=0 \text{ and } y=1$
\end{enumerate}

\item Which of the following functions is a probability density function of a random variable $X$?
\begin{enumerate}
\item $f\brak{x} =
\begin{cases}
x\brak{2-x} & 0<x<2 \\ 0 & \text{elsewhere}
\end{cases}$
\item $f\brak{x} =
\begin{cases}
x\brak{1-x} & 0<x<1 \\ 0 & \text{elsewhere}
\end{cases}$
\item $f\brak{x} =
\begin{cases}
2xe^{-x^2} & -1<x<1 \\ 0 & \text{elsewhere}
\end{cases}$
\item $f\brak{x} =
\begin{cases}
2xe^{-x^2} & x>0 \\ 0 & \text{elsewhere}
\end{cases}$
\end{enumerate}

\item Let $X_1,X_2,X_3 \text{ and } X_4$ be independent standard normal random variables. The distribution of
\begin{align*}
W = \frac{1}{2}\cbrak{\brak{X_1-X_2}^2 + \brak{X_3-X_4}^2}
\end{align*} is
\begin{enumerate}
\item $N\brak{0,1}$
\item $N\brak{0,2}$
\item $\chi_{2}^{2}$
\item $\chi_{4}^{2}$
\end{enumerate}

\item For $n\geq 1$, let $\cbrak{X_n}$ be a sequence of independent random variables with
\begin{align*}
	P\brak{X_n = n} = P\brak{X_n = -n} = \frac{1}{2n^2}, P\brak{X_n = 0} = 1 - \frac{1}{n^2}.
\end{align*}
Then, which of the following statements is \textbf{TRUE} for the sequence $\cbrak{X_n}$?
\begin{enumerate}
\item Weak Law of Large Numbers holds but Strong Law of Large Numbers does not hold
\item Weak Law of Large Numbers does not hold but Strong Law of Large Numbers holds
\item Both Weak Law of Large Numbers and Strong Law of Large Numbers hold
\item Both Weak Law of Large Numbers and Strong Law of Large Numbers do not hold
\end{enumerate}

\end{enumerate}
\end{document}
