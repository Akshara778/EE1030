%iffalse
\let\negmedspace\undefined
\let\negthickspace\undefined
\documentclass[journal,12pt,onecolumn]{IEEEtran}
\usepackage{cite}
\usepackage{amsmath,amssymb,amsfonts,amsthm}
\usepackage{algorithmic}
\usepackage{graphicx}
\usepackage{textcomp}
\usepackage{xcolor}
\usepackage{txfonts}
\usepackage{listings}
\usepackage{enumitem}
\usepackage{mathtools}
\usepackage{gensymb}
\usepackage{comment}
\usepackage[breaklinks=true]{hyperref}
\usepackage{tkz-euclide} 
\usepackage{listings}
\usepackage{gvv}
\usepackage{circuitikz}
%\def\inputGnumericTable{}                         
\usepackage[latin1]{inputenc}                                
\usepackage{color}                                            
\usepackage{array}                                            
\usepackage{longtable}                                       
\usepackage{calc}                                             
\usepackage{multirow}                                         
\usepackage{hhline}                                           
\usepackage{ifthen}                                           
\usepackage{lscape}
\usepackage{tabularx}
\usepackage{array}
\usepackage{float}
\usepackage{multicol}

\newtheorem{theorem}{Theorem}[section]
\newtheorem{problem}{Problem}
\newtheorem{proposition}{Proposition}[section]
\newtheorem{lemma}{Lemma}[section]
\newtheorem{corollary}[theorem]{Corollary}
\newtheorem{example}{Example}[section]
\newtheorem{definition}[problem]{Definition}
\newcommand{\BEQA}{\begin{eqnarray}}
\newcommand{\EEQA}{\end{eqnarray}}
\newcommand{\define}{\stackrel{\triangle}{=}}
\theoremstyle{remark}
\newtheorem{rem}{Remark}

\begin{document}
\bibliographystyle{IEEEtran}
\title{2016-EE-53-65}
\author{EE24BTECH11003 - Akshara Sarma Chennubhatla}
\maketitle
\begin{enumerate}

\item A single-phase thyristor-bridge rectifier is fed from a $230 V, 50$ Hz, single-phase AC mains. If it is delivering a constant DC current of $10$ A, at firing angle of $30^\circ$, then value of the power factor at AC mains is
\hfill{\brak{2016}}
\begin{enumerate}
\item $0.87$
\item $0.9$
\item $0.78$
\item $0.45$
\end{enumerate}

\item The switches $T1$ and $T2$ in Figure $\brak{a}$ are switched in a complementary fashion with sinusoidal pulse width modulation technique. The modulating voltage $v_m(t) = 0.8 \sin\brak{200\pi t}$ V and the triangular carrier voltage $\brak{v_c}$ are as shown in Figure $\brak{b}$. The carrier frequency is $5$ kHz. The peak value of the $100$ Hz component of the load current $\brak{i_L}$, in ampere, is $\rule{2cm}{0.1pt}.$
\hfill{\brak{2016}}
\begin{center}
\begin{circuitikz}
\tikzstyle{every node}=[font=\large]
\draw  (4,15) rectangle (8,14.25);
\draw  (4,15) -- (8,15) -- (8.5,15.25) -- (4.5,15.25) -- cycle;
\draw [short] (8,14.25) -- (8.5,14.5);
\draw [short] (8.5,15.25) -- (8.5,14.5);
\draw [short] (4,14.5) -- (3.75,14.75);
\draw [short] (4,14.75) -- (3.75,15);
\draw [short] (4,14.5) -- (3.75,14.75);
\draw [short] (4,14.25) -- (3.75,14.5);
\draw [short] (4,15) -- (3.75,15.25);
\draw  (11.75,14.5) rectangle (15.75,15.25);
\draw  (11.75,15.25) -- (15.75,15.25) -- (16.25,15.5) -- (12.25,15.5) -- cycle;
\draw [short] (15.75,14.5) -- (16.25,14.75);
\draw [short] (16.25,15.5) -- (16.25,14.75);
\draw [->, >=Stealth] (8.25,14.75) -- (9.75,14.75);
\draw [->, >=Stealth] (17.5,15) -- (16,15);
\draw [short] (11.75,14.5) -- (11.5,14.75);
\draw [short] (11.75,14.75) -- (11.5,15);
\draw [short] (11.75,15) -- (11.5,15.25);
\draw [short] (11.75,15.25) -- (11.5,15.5);
\draw [short] (4,12) -- (8.5,12);
\draw [short] (4,11.25) -- (8.5,11.25);
\draw [short] (8.5,12) .. controls (8.5,12) and (8.25,11.5) .. (8.5,11.25);
\draw [short] (4,12) .. controls (3.75,11.75) and (3.75,11.25) .. (4,11.25);
\draw  (12,12) rectangle (16,11.25);
\draw  (12,12) -- (16,12) -- (16.5,12.25) -- (12.5,12.25) -- cycle;
\draw [short] (16,11.25) -- (16.5,11.5);
\draw [short] (16.5,12.25) -- (16.5,11.5);
\draw [short] (12,11.25) -- (11.75,11.5);
\draw [short] (12,11.5) -- (11.75,11.75);
\draw [short] (12,11.75) -- (11.75,12);
\draw [short] (12,12) -- (11.75,12.25);
\draw [->, >=Stealth] (16.5,11) .. controls (17.25,11.5) and (17,12.5) .. (16.25,12.5) ;
\draw [->, >=Stealth] (8.25,11) .. controls (9.5,11) and (8.25,13.25) .. (8,11.75) ;
\draw [short] (8.5,12) .. controls (8.75,11.75) and (8.75,11.25) .. (8.5,11.25);
\draw [short] (4,11.75) -- (3.75,12);
\draw [short] (4,11.5) -- (3.75,11.75);
\draw [short] (4,11.25) -- (3.75,11.5);
\node [font=\Large] at (6.25,15.75) {Tensile load};
\node [font=\large] at (9.25,14) {80 kN};
\node [font=\Large] at (14.25,16) {Compressive load};
\node [font=\large] at (17,14.25) {80 kN};
\node [font=\large] at (9.75,11.75) {64$\pi$ Nm};
\node [font=\large] at (17.75,12.5) {320 Nm};
\node [font=\Large] at (14.25,12.75) {Bending load};
\node [font=\Large] at (6.25,12.5) {Torsional load};
\end{circuitikz}

\end{center}

\item The voltage $\brak{v_s}$ across and the current $\brak{i_s}$ through a semiconductor switch during a turn-ON transition are shown in figure. The energy dissipated during the turn-ON transition, in mJ, is $\rule{2cm}{0.1pt}.$
\hfill{\brak{2016}}
\begin{center}
\begin{circuitikz}
\tikzstyle{every node}=[font=\large]
\draw [short] (11,18) -- (11,8.25);
\draw [->, >=Stealth] (9.25,10.25) -- (20.75,10.25);
\draw [->, >=Stealth] (9.25,14.75) -- (20.75,14.75);
\draw [line width=1.5pt, short] (11,10.25) -- (13,10.25);
\draw [line width=1.5pt, short] (13,10.25) -- (15,12.75);
\draw [line width=1.5pt, short] (15,12.75) -- (15,11.5);
\draw [line width=1.5pt, short] (15,11.5) -- (20.25,11.5);
\draw [line width=1.5pt, short] (11,17) -- (15,17);
\draw [line width=1.5pt, short] (15,17) -- (17,14.75);
\draw [line width=1.5pt, short] (17,14.75) -- (20.25,14.75);
\draw [dashed] (13,8) -- (13,18.75);
\draw [dashed] (15,18.75) -- (15,8);
\draw [dashed] (17,18.75) -- (17,8);
\draw [->, >=Stealth] (11.5,14.75) -- (11.5,17);
\draw [->, >=Stealth] (10.75,15.25) -- (10.75,16.5);
\draw [->, >=Stealth] (10.75,10.75) -- (10.75,12.5);
\draw [->, >=Stealth] (15.5,11.5) -- (15.5,12.75);
\draw [->, >=Stealth] (17.5,10.25) -- (17.5,11.5);
\draw [short] (15.25,12.75) -- (15.75,12.75);
\draw [<->, >=Stealth] (13,8.5) -- (15,8.5);
\draw [<->, >=Stealth] (15,8.5) -- (17,8.5);
\node [font=\large] at (10.25,11.25) {$i_s$};
\node [font=\large] at (10.25,15.75) {$v_s$};
\node [font=\large] at (12.25,15.75) {600 V};
\node [font=\large] at (16,12.25) {50 A};
\node [font=\large] at (18.25,10.75) {100 A};
\node [font=\large] at (14,7.75) {T1 = 1\mus};
\node [font=\large] at (16,7.75) {T2 = 1\mus};
\node [font=\large] at (21,14.75) {t};
\node [font=\large] at (21,10.25) {t};
\end{circuitikz}

\end{center}

\item A single-phase $400$ V, $50$ Hz transformer has an iron loss of $5000$ W at the rated condition. When operated at $200$ V, $25$ Hz, the iron loss is $2000$ W. When operated at $416$ V, $52$ Hz, the value of the hysteresis loss divided by the eddy current loss is $\rule{2cm}{0.1pt}.$
\hfill{\brak{2016}}

\item A DC shunt generator delivers $45$ A at a terminal voltage of $220$ V. The armature and the shunt field resistances are $0.01 \Omega$  and $44 \Omega$ respectively. The stray losses are $375$ W. The percentage efficiency of the DC generator is $\rule{2cm}{0.1pt}.$
\hfill{\brak{2016}}

\item A three-phase, $50$ Hz salient-pole synchronous motor has a per-phase direct-axis reactance $\brak{X_d}$ of $0.8$ pu and a per-phase quadrature-axis reactance $\brak{X_q}$ of $0.6$ pu. Resistance of the machine is negligible. It is drawing full-load current at $0.8$ pf $\brak{\text{leading}}$. When the terminal voltage is $1$ pu, per-phase induced voltage, in pu, is $\rule{2cm}{0.1pt}.$
\hfill{\brak{2016}}

\item A single-phase, $22$ kVA, $2200$ V/ $220$ V, $50$ Hz, distribution transformer is to be connected as an auto-transformer to get an output voltage of $2420$ V. Its maximum kVA rating as an auto-transformer is
\hfill{\brak{2016}}
\begin{enumerate}
\item $0.87$
\item $0.9$
\item $0.78$
\item $0.45$
\end{enumerate}

\item A single-phase full-bridge voltage source inverter $\brak{\text{VSI}}$ is fed from a $300$ V battery. A pulse of $120^\circ$ duration is used to trigger the appropriate devices in each half-cycle. The rms value of the fundamental component of the output voltage, in volts, is
\hfill{\brak{2016}}
\begin{enumerate}
\item $234$
\item $245$
\item $300$
\item $331$
\end{enumerate}

\item A single-phase transmission line has two conductors each of $10$ mm radius. These are fixed at a center-to-center distance of $1$ m in a horizontal plane. This is now converted to a three-phase transmission line by introducing a third conductor of the same radius. This conductor is fixed at an equal distance $D$ from the two single-phase conductors. The three-phase line is fully transposed. The positive sequence inductance per phase of the three-phase system is to be $5\%$ more than that of the inductance per conductor of the single-phase system. The distance $D$, in meters, is $\rule{2cm}{0.1pt}.$
\hfill{\brak{2016}}

\item In the circuit shown below, the supply voltage is $10\sin\brak{1000t}$ volts. The peak value of the steady state current through the $1$ Ω resistor, in amperes, is $\rule{2cm}{0.1pt}.$
\hfill{\brak{2016}}
\begin{center}
\begin{center}
\begin{circuitikz}
\tikzstyle{every node}=[font=\LARGE]
\draw [short] (12.25,15.75) .. controls (12,6.5) and (19,9.25) .. (19.5,15);
\draw [short] (12.75,15.75) .. controls (13,6.25) and (19,11.25) .. (19,15);
\draw  (12.5,15.75) ellipse (0.3cm and 0.2cm);
\draw  (19.25,15) ellipse (0.3cm and 0.2cm);
\draw [line width=1.3pt, short] (19.55,14.9) -- (9.5,10.75);
\draw [line width=1.3pt, short] (12.25,15.75) -- (9,11.25);
\draw [ line width=2pt , rotate around={-288:(9.25,11)}] (9.25,11) ellipse (0.25cm and 0.5cm);
\draw [line width=1pt, dashed] (12.5,15.75) -- (19.25,15);
\draw [line width=1.2pt, dashed] (15.5,16.25) -- (15,6.75);
\draw [dashed] (8.5,10.8) -- (15.5,15.5);
\draw [short] (13.25,10.75) .. controls (14.25,9) and (14.25,9.25) .. (14.25,7);
\draw [short] (17,10.3) .. controls (15.75,9) and (15.75,9) .. (15.5,7);
\draw [short] (14.25,7) .. controls (15,6.5) and (15,6.75) .. (15.5,7);
\draw [->, >=Stealth] (15.25,15.5) -- (9.25,11.5);
\draw [->, >=Stealth] (9.5,11.25) -- (11,12.25);
\node [font=\large] at (11,11.75) {$F_x$};
\node [font=\Large] at (11.25,13.25) {x};
\draw [->, >=Stealth] (10.75,10.25) -- (9.25,11);
\node [font=\large] at (11.25,10) {Stone of mass 0.1 kg};
\draw [short] (9.25,11) -- (9.25,12);
\end{circuitikz}
\end{center}

\end{center}

\item A dc voltage with ripple is given by $v\brak{t} = \sbrak{100 + 10\sin\brak{\omega t} - 5\sin\brak{3\omega t}}$ volts. Measurements of this voltage $v\brak{t}$, made by moving-coil and moving-iron voltmeters, show readings of $V_1$ and $V_2$ respectively. The value of $V_2 − V_1$ , in volts, is $\rule{2cm}{0.1pt}.$
\hfill{\brak{2016}}

\item The circuit below is excited by a sinusoidal source. The value of $R$, in $\ohm$, for which the admittance of the circuit becomes a pure conductance at all frequencies is $\rule{2cm}{0.1pt}.$
\hfill{\brak{2016}}
\begin{center}
\begin{circuitikz}
\tikzstyle{every node}=[font=\large]
\draw [short] (9.5,15.5) -- (13.75,15.5);
\draw [short] (9.75,15.5) -- (10.25,16);
\draw [short] (10.25,15.5) -- (10.75,16);
\draw [short] (10.75,15.5) -- (11.25,16);
\draw [short] (11.25,15.5) -- (11.75,16);
\draw [short] (11.75,15.5) -- (12.25,16);
\draw [short] (12.25,15.5) -- (12.75,16);
\draw [short] (12.75,15.5) -- (13.25,16);
\draw [short] (13.25,15.5) -- (13.75,16);
\draw [short] (11.75,15.5) -- (13.75,15.5);
\draw [short] (10,15.5) -- (10.75,12);
\draw [short] (13,15.5) -- (12.25,12);
\draw [short] (10.75,12) -- (12.25,12);
\draw [short] (10.75,12) -- (10.75,9.5);
\draw [short] (12.25,12) -- (12.25,9.5);
\draw [short] (10.75,9.5) -- (12.25,9.5);
\draw [<->, >=Stealth] (8.75,15.5) -- (8.75,11.75);
\draw [<->, >=Stealth] (8.75,11.75) -- (8.75,9.5);
\draw [line width=1.6pt, ->, >=Stealth] (11.5,9.5) -- (11.5,8.25);
\draw [<->, >=Stealth] (10,16.75) -- (13,16.75);
\draw [<->, >=Stealth] (10.75,11.5) -- (12.25,11.5);
\draw [short] (8.5,11.75) -- (9,11.75);
\draw [short] (8.5,15.5) -- (9,15.5);
\draw [short] (8.5,9.5) -- (9,9.5);
\draw [short] (10,17) -- (10,16.25);
\draw [short] (13,17) -- (13,16.25);
\node [font=\large] at (11.5,17.25) {$d_1 = 20 mm$};
\node [font=\large] at (9.25,13.5) {2 m};
\node [font=\large] at (9.5,10.5) {1.5 m};
\node [font=\large] at (13.5,11.5) {$d_2 = 10 mm$};
\node [font=\large] at (12.5,9.5) {A};
\node [font=\large] at (11.75,7.75) {P = 30$\pi$ kN};
\end{circuitikz}

\end{center}

\item In the circuit shown below, the node voltage $V_A$ is $\rule{2cm}{0.1pt}$ V.
\hfill{\brak{2016}}
\begin{center}
\begin{circuitikz}
\tikzstyle{every node}=[font=\normalsize]
\draw [ color={rgb,255:red,242; green,133; blue,0} ] (5,14.75) rectangle (17.5,14.5);
\draw [ color={rgb,255:red,242; green,133; blue,0} ] (6,14.5) rectangle (6.5,14.25);
\draw [ color={rgb,255:red,242; green,133; blue,0} ] (6.25,14.5) rectangle (6.75,14.25);
\draw [ color={rgb,255:red,242; green,133; blue,0} ] (9.75,14.5) rectangle (10.25,14.25);
\draw [ color={rgb,255:red,242; green,133; blue,0} ] (10,14.5) rectangle (10.5,14.25);
\draw [ color={rgb,255:red,0; green,84; blue,194}, short] (5,14.75) -- (5,16);
\draw [ color={rgb,255:red,0; green,84; blue,194}, short] (5,16) -- (5.25,16);
\draw [ color={rgb,255:red,0; green,84; blue,194}, short] (5.25,16) -- (5.5,15);
\draw [ color={rgb,255:red,0; green,84; blue,194}, short] (5.5,15) -- (11.25,15);
\draw [ color={rgb,255:red,0; green,84; blue,194}, short] (11.25,15) -- (11.5,16);
\draw [ color={rgb,255:red,0; green,84; blue,194}, short] (11.5,16) -- (11.75,16);
\draw [ color={rgb,255:red,0; green,84; blue,194}, short] (11.75,16) -- (11.75,14.75);
\draw [ color={rgb,255:red,242; green,133; blue,0} ] (6,15) rectangle (6.75,15.25);
\draw [ color={rgb,255:red,242; green,133; blue,0} ] (6.5,15) rectangle (6.25,15.5);
\draw [ color={rgb,255:red,242; green,133; blue,0} ] (9.75,15) rectangle (10.5,15.25);
\draw [ color={rgb,255:red,242; green,133; blue,0} ] (10,15) rectangle (10.25,15.5);
\draw [ color={rgb,255:red,242; green,133; blue,0} ] (5,12.25) rectangle (17.5,6);
\draw [ color={rgb,255:red,0; green,84; blue,194}, short] (5,12.25) -- (5,13);
\draw [ color={rgb,255:red,0; green,84; blue,194}, short] (11.75,12.25) -- (11.75,13);
\draw [ color={rgb,255:red,0; green,84; blue,194}, dashed] (6.25,15.75) -- (6.25,5.5);
\draw [ color={rgb,255:red,0; green,84; blue,194}, dashed] (10,15.75) -- (10,5.5);
\draw [ color={rgb,255:red,242; green,133; blue,0}, short] (6.25,11.25) -- (6,11.25);
\draw [ color={rgb,255:red,242; green,133; blue,0}, short] (6.25,11.25) -- (6.5,11.25);
\draw [ color={rgb,255:red,242; green,133; blue,0}, short] (6.5,11.25) -- (6.75,11);
\draw [ color={rgb,255:red,242; green,133; blue,0}, short] (6.75,11) -- (6.5,10.75);
\draw [ color={rgb,255:red,242; green,133; blue,0}, short] (6.5,10.75) -- (6,10.75);
\draw [ color={rgb,255:red,242; green,133; blue,0}, short] (6,10.75) -- (5.75,11);
\draw [ color={rgb,255:red,242; green,133; blue,0}, short] (5.75,11) -- (6,11.25);
\draw [ color={rgb,255:red,242; green,133; blue,0}, short] (9.75,11.25) -- (10.25,11.25);
\draw [ color={rgb,255:red,242; green,133; blue,0}, short] (10.25,11.25) -- (10.5,11);
\draw [ color={rgb,255:red,242; green,133; blue,0}, short] (10.5,11) -- (10.25,10.75);
\draw [ color={rgb,255:red,242; green,133; blue,0}, short] (10.25,10.75) -- (9.75,10.75);
\draw [ color={rgb,255:red,242; green,133; blue,0}, short] (9.75,10.75) -- (9.5,11);
\draw [ color={rgb,255:red,242; green,133; blue,0}, short] (9.5,11) -- (9.75,11.25);
\draw [ color={rgb,255:red,242; green,133; blue,0}, short] (6,7.5) -- (6.5,7.5);
\draw [ color={rgb,255:red,242; green,133; blue,0}, short] (6.5,7.5) -- (6.75,7.25);
\draw [ color={rgb,255:red,242; green,133; blue,0}, short] (6.75,7.25) -- (6.5,7);
\draw [ color={rgb,255:red,242; green,133; blue,0}, short] (6.5,7) -- (6,7);
\draw [ color={rgb,255:red,242; green,133; blue,0}, short] (6,7) -- (5.75,7.25);
\draw [ color={rgb,255:red,242; green,133; blue,0}, short] (5.75,7.25) -- (6,7.5);
\draw [ color={rgb,255:red,242; green,133; blue,0}, short] (9.75,7.5) -- (10.25,7.5);
\draw [ color={rgb,255:red,242; green,133; blue,0}, short] (10.25,7.5) -- (10.5,7.25);
\draw [ color={rgb,255:red,242; green,133; blue,0}, short] (10.5,7.25) -- (10.25,7);
\draw [ color={rgb,255:red,242; green,133; blue,0}, short] (10.25,7) -- (9.75,7);
\draw [ color={rgb,255:red,242; green,133; blue,0}, short] (9.75,7) -- (9.5,7.25);
\draw [ color={rgb,255:red,242; green,133; blue,0}, short] (9.5,7.25) -- (9.75,7.5);
\draw [ color={rgb,255:red,0; green,84; blue,194}, dashed] (5.5,11) -- (13.75,11);
\draw [ color={rgb,255:red,0; green,84; blue,194}, dashed] (5.5,7.25) -- (13.75,7.25);
\draw [ color={rgb,255:red,0; green,84; blue,194}, dashed] (6.75,9.25) -- (18,9.25);
\draw [ color={rgb,255:red,0; green,84; blue,194}, dashed] (8.25,5.75) -- (10.25,5.75);
\draw [ color={rgb,255:red,0; green,84; blue,194}, dashed] (9.25,15.25) -- (11.25,15.25);
\draw [short] (5,13) .. controls (5.5,13.25) and (5.75,13) .. (6.25,13);
\draw [short] (6.25,13) .. controls (7,13.25) and (7.25,13.25) .. (8,13.25);
\draw [short] (8,13.25) .. controls (8.75,13) and (9,13) .. (10,13);
\draw [short] (10,13) -- (11.75,13.25);
\draw [short] (5,5.5) .. controls (5.75,5.25) and (5.5,5.25) .. (6.25,5.25);
\draw [short] (6.25,5.25) .. controls (7,5) and (7,5.5) .. (8,5.5);
\draw [short] (8,5.5) -- (10,5.5);
\draw [ color={rgb,255:red,0; green,84; blue,194}, short] (10,5.5) -- (11.25,5.5);
\draw [ color={rgb,255:red,0; green,84; blue, 194}, short] (11.25,5.5) -- (11.25,6);
\draw [ color={rgb,255:red,0; green,84; blue, 194}, short] (5,5.5) -- (5,6);
\draw [ color={rgb,255:red,0; green,84; blue, 194}, short] (11.75,13.25) -- (11.75,13);
\draw [ color={rgb,255:red,0; green,84; blue, 194}, <->, >=Stealth] (5,4.75) -- (8,4.75);
\draw [ color={rgb,255:red,0; green,84; blue, 194}, <->, >=Stealth] (8.25,4.75) -- (11.25,4.75);
\draw [ color={rgb,255:red,0; green,84; blue, 194}, <->, >=Stealth] (11.25,4.75) -- (18.75,4.75);
\draw [ color={rgb,255:red,0; green,84; blue, 194}, <->, >=Stealth] (18,12.25) -- (18,6);
\node [font=\normalsize, color={rgb,255:red,0; green,84; blue, 194}] at (18.25,9.5) {200};
\node [font=\normalsize] at (11,10.25) {50};
\node [font=\normalsize] at (11,8.25) {50};
\node [font=\normalsize] at (5.75,11.5) {A};
\node [font=\normalsize] at (9.5,11.5) {B};
\node [font=\normalsize] at (8.25,9.5) {O};
\node [font=\normalsize] at (5.75,7.75) {C};
\node [font=\normalsize] at (9.5,7.75) {D};
\node [font=\normalsize] at (6.25,4.5) {100};
\node [font=\normalsize] at (9.75,4.5) {100};
\node [font=\normalsize] at (14.5,4.25) {300};
\node [font=\normalsize] at (15.25,13) {F=10kN};
\node [font=\normalsize] at (8.5,16.75) {200};
\node [font=\normalsize] at (12.75,15.75) {Steel channel};
\node [font=\normalsize] at (16.25,16.75) {Not to scale};
\node [font=\normalsize] at (16.25,16.25) {Dimensions in mm};
\draw [ color={rgb,255:red,0; green,84; blue,194}, <->, >=Stealth] (5.25,16.5) -- (11.5,16.5);
\draw [ color={rgb,255:red,0; green,84; blue,194}, ->, >=Stealth] (16.25,14) -- (16.25,12.25);
\draw [ color={rgb,255:red,0; green,84; blue,194}, <->, >=Stealth] (10.75,11) -- (10.75,9.25);
\draw [ color={rgb,255:red,0; green,84; blue,194}, <->, >=Stealth] (10.75,9.25) -- (10.75,7.25);
\end{circuitikz}

\end{center}


\end{enumerate}
\end{document}
