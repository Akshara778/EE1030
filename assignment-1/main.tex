\let\negmedspace\undefined
\let\negthickspace\undefined
\documentclass[journal,12pt,twocolumn,article]{IEEEtran}
\usepackage{cite}
\usepackage{color,soul}
\usepackage{amsmath,amssymb,amsfonts,amsthm}
\usepackage{algorithmic}
\usepackage{graphicx}
\usepackage{textcomp}
\usepackage{xcolor}
\usepackage{txfonts}
\usepackage{listings}
\usepackage{enumitem}
\usepackage{mathtools}
\usepackage{gensymb}
\usepackage{comment}
\usepackage[breaklinks=true]{hyperref}
\usepackage{tkz-euclide} 
\usepackage{listings}
\usepackage{multicol}
\usepackage{gvv}       
\usepackage[dvipsnames]{xcolor}
\def\inputGnumericTable{}                                
\usepackage[latin1]{inputenc}                            
\usepackage{color}                                       
\usepackage{array}                                       
\usepackage{longtable}                                   
\usepackage{calc}                                        
\usepackage{multirow}                                    
\usepackage{hhline}                                      
\usepackage{ifthen}                                      
\usepackage{lscape}
\newtheorem{theorem}{Theorem}[section]
\newtheorem{problem}{Problem}
\newtheorem{proposition}{Proposition}[section]
\newtheorem{lemma}{Lemma}[section]
\newtheorem{corollary}[theorem]{Corollary}
\newtheorem{example}{Example}[section]
\newtheorem{definition}[problem]{Definition}
\newcommand{\BEQA}{\begin{eqnarray}}
\newcommand{\EEQA}{\end{eqnarray}}
\newcommand{\define}{\stackrel{\triangle}{=}}
\theoremstyle{remark}
\newtheorem{rem}{Remark}
\begin{document}
\begin{enumerate}[start = 20]
\bibliographystyle{IEEEtran}
\vspace{3cm}
\title{ASSIGNMENT 1}
\author{EE24BTECH11003 - Akshara Sarma Chennubhatla}
\maketitle
\newpage
\bigskip
\section*{C: MCQs With One Correct Answer}
\item If $\alpha + \beta$ = $\frac{\pi}{2}$ and $\beta + \gamma$ = $\alpha$, then $\tan \alpha$ equals
\hfill\brak{2001S}
\begin{enumerate}
\begin{multicols}{2}
\item $2\brak{\tan \beta + \tan \gamma}$
\columnbreak
\item $\tan\beta$ + $\tan\gamma$
\end{multicols}
\begin{multicols}{2}
\item $\tan\beta$ + 2 $\tan\gamma$
\item 2 $\tan\beta$ + $\tan\gamma$
\end{multicols}
\end{enumerate}
\item The number of integral values of $k$ for which the equation $7\cos x + 5\sin x = 2k+1$ has a solution is
\hfill\brak{2002S}
\begin{enumerate}
\begin{multicols}{2}
\item 4
\columnbreak
\item 8
\end{multicols}
\begin{multicols}{2}
\item 10
\columnbreak
\item 12 
\end{multicols}
\end{enumerate}
\item Given both $\theta$ and $\phi$ are acute angles and $\sin\theta$ = $\frac{1}{2}$, $\cos\phi = \frac{1}{3}$, then the value of $\theta + \phi$ belongs to
\hfill\brak{2004S}
\begin{enumerate}
\begin{multicols}{2}
\item $(\frac{\pi}{3},\frac{\pi}{2}]$
\columnbreak
\item $\brak{\frac{\pi}{2},\frac{2\pi}{3}}$
\end{multicols}
\begin{multicols}{2}
\item $(\frac{2\pi}{3},\frac{5\pi}{6}]$
\columnbreak
\item $(\frac{5\pi}{6},\pi]$
\end{multicols}
\end{enumerate}
\item $\cos\brak{\alpha - \beta}=1$ and $\cos\brak{\alpha + \beta}=\frac{1}{e}$ where $\alpha, \beta \in \sbrak{-\pi,\pi}$. Pairs of $\alpha, \beta$ which satisfy both the equations is/are
\hfill\brak{2005S}
\begin{enumerate}
\begin{multicols}{2}
\item 0
\columnbreak
\item 1
\end{multicols}
\begin{multicols}{2}
\item 2
\columnbreak
\item 4 
\end{multicols}
\end{enumerate}
\item The values of $\theta \in \brak{0,2\pi}$ for which $2\sin^2\theta - 5\sin\theta + 2 > 0$, are
\hfill\brak{2006-3M,-1}
\begin{enumerate}
\begin{multicols}{2}
\item $\brak{0,\frac{\pi}{6}}\cup\brak{\frac{5\pi}{6},2\pi}$
\columnbreak
\item $\brak{\frac{\pi}{8},\frac{5\pi}{6}}$
\end{multicols}
\begin{multicols}{2}
\item $\brak{0,\frac{\pi}{8}}\cup\brak{\frac{\pi}{6},\frac{5\pi}{6}}$
\columnbreak
\item $\brak{\frac{41\pi}{48},\pi}$
\end{multicols}
\end{enumerate}
\item Let $\theta \in \brak{0,\frac{\pi}{4}}$ and 
\begin{align*}
t_1 = \brak{\tan\theta}^{\tan\theta}, t_2 = \brak{\tan\theta}^{cot\theta},\\t_3 = \brak{cot\theta}^{\tan\theta}, t_4 = \brak{cot\theta}^{cot\theta},
\end{align*} then
\hfill\brak{2006-3M,-1}
\begin{enumerate}
\begin{multicols}{2}
\item $t_1>t_2>t_3>t_4$
\columnbreak
\item $t_4>t_3>t_1>t_2$
\end{multicols}
\begin{multicols}{2}
\item $t_3>t_1>t_2>t_4$
\columnbreak
\item $t_2>t_3>t_1>t_4$
\end{multicols}
\end{enumerate}
\item The number of solutions of the pair of equations
\begin{align*}
2\sin^2\theta - \cos2\theta = 0\\
2\cos^2\theta - 3\sin\theta = 0
\end{align*}
in the interval $\sbrak{0,2\pi}$ is
\hfill\brak{2007-3 Marks}
\begin{enumerate}
\begin{multicols}{2}
\item zero
\columnbreak
\item one
\end{multicols}
\begin{multicols}{2}
\item two
\columnbreak
\item four
\end{multicols}
\end{enumerate}
\item For $x \in \brak{0,\pi}$, the equation $\sin x + 2\sin 2x - \sin 3x = 3$ has
\hfill\brak{JEE Adv. 2014}
\begin{enumerate}
\item infinitely many solutions
\item three solutions
\item one solution
\item no solution
\end{enumerate}
\item Let $S = \cbrak{x \in \brak{-\pi,\pi} : x \neq 0, \pm \frac{\pi}{2}}$. The sum of all distinct solutions of the equation $\sqrt{3} \sec x + \cosec x + 2\brak{\tan x - \cot x} = 0$ in the set S is equal to
\hfill\brak{JEE Adv. 2016}
\begin{enumerate}
\begin{multicols}{2}
\item $-\frac{7\pi}{9}$
\columnbreak
\item $-\frac{2\pi}{9}$
\end{multicols}
\begin{multicols}{2}
\item 0
\columnbreak
\item $\frac{5\pi}{9}$
\end{multicols}
\end{enumerate}
\item The value of 
\begin{align*}
\sum_{k=1}^{13} \frac{1}{\sin\brak{\frac{\pi}{4} + \frac{\brak{k-1}\pi}{6}}\sin\brak{\frac{\pi}{4} + \frac{k\pi}{6}}}
\end{align*}
is equal to
\hfill\brak{JEE Adv. 2016}
\begin{enumerate}
\begin{multicols}{2}
\item $3-\sqrt{3}$
\columnbreak
\item $2\brak{3-\sqrt{3}}$
\end{multicols}
\begin{multicols}{2}
\item $2\brak{\sqrt{3}-1}$
\columnbreak
\item $2\brak{2-\sqrt{3}}$
\end{multicols}
\end{enumerate}
\end{enumerate}
\section*{D: MCQs with One or More than One Correct}
\begin{enumerate}
\item 
\begin{multline*}
\brak{1 + \cos\frac{\pi}{8}}\brak{1 + \cos\frac{3\pi}{8}}\\
\brak{1 + \cos\frac{5\pi}{8}}\brak{1 + \cos\frac{7\pi}{8}} 
\end{multline*}
is equal to
\hfill\brak{1984-3 Marks}
\begin{enumerate}
\begin{multicols}{2}
\item $\frac{1}{2}$
\columnbreak
\item $\cos \frac{\pi}{8}$
\end{multicols}
\begin{multicols}{2}
\item $\frac{1}{8}$
\columnbreak
\item $\frac{1+\sqrt{2}}{2\sqrt{2}}$
\end{multicols}
\end{enumerate}
\item The expression 
\begin{align*}
3\sbrak{\sin^4\brak{\frac{3\pi}{2} - \alpha} + \sin^4\brak{3\pi + \alpha}}  \\ - 2\sbrak{\sin^6\brak{\frac{\pi}{2} + \alpha} + \sin^6\brak{5\pi - \alpha}}
\end{align*}
is equal to
\hfill\brak{1986-2 Marks}
\begin{enumerate}
\begin{multicols}{2}
\item 0
\columnbreak
\item 1
\end{multicols}
\begin{multicols}{2}
\item 3
\columnbreak
\item $\sin4\alpha + \cos6\alpha$
\end{multicols}
\begin{multicols}{2}
\item none of these
\end{multicols}
\end{enumerate}
\item The number of all possible triplets $\brak{a_1, a_2, a_3}$ such that $a_1 + a_2 \cos\brak{2x} + a_3\sin^2\brak{x} = 0$ for all $x$ is
\hfill\brak{1987-2 Marks}
\begin{enumerate}
\begin{multicols}{2}
\item zero
\columnbreak
\item one
\end{multicols}
\begin{multicols}{2}
\item three
\columnbreak
\item infinite
\end{multicols}
\begin{multicols}{2}
\item none
\end{multicols}
\end{enumerate}
\item The values of $\theta$ lying between $\theta = 0$ and $\theta = \frac{\pi}{2}$ and satisfying the equation
\[\begin{vmatrix}
$$1+\sin^2\theta$$ & $$\cos^2\theta$$ & $$4\sin4\theta$$\\
$$\sin^2\theta$$ & $$1+\cos^2\theta$$ & $$4\sin4\theta$$\\
$$\sin^2\theta$$ & $$\cos^2\theta$$ & $$1+4\sin4\theta$$
\end{vmatrix} = 0\] are
\hfill\brak{1988-2 Marks}
\begin{enumerate}
\begin{multicols}{2}
\item $\frac{7\pi}{24}$
\columnbreak
\item $\frac{5\pi}{24}$
\end{multicols}
\begin{multicols}{2}
\item $\frac{11\pi}{24}$
\columnbreak
\item $\frac{\pi}{24}$
\end{multicols}
\end{enumerate}
\item Let $2\sin^2x+3\sin x-2>0$ and $x^2-x-2<0 \brak{x \text{is measured in radians}}$. Then $x$ lies in the interval
\hfill\brak{1994}
\begin{enumerate}
\begin{multicols}{2}
\item $\brak{\frac{\pi}{6},\frac{5\pi}{6}}$
\columnbreak
\item $\brak{-1,\frac{5\pi}{6}}$
\end{multicols}
\begin{multicols}{2}
\item $\brak{-1,2}$
\columnbreak
\item $\brak{\frac{\pi}{6},2}$
\end{multicols}
\end{enumerate}
\end{enumerate}
\end{document}
