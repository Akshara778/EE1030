\let\negmedspace\undefined
\let\negthickspace\undefined
\documentclass[journal,12pt,twocolumn,article]{IEEEtran}
\usepackage{cite}
\usepackage{color,soul}
\usepackage{amsmath,amssymb,amsfonts,amsthm}
\usepackage{algorithmic}
\usepackage{graphicx}
\usepackage{textcomp}
\usepackage{xcolor}
\usepackage{txfonts}
\usepackage{listings}
\usepackage{enumitem}
\usepackage{mathtools}
\usepackage{gensymb}
\usepackage{comment}
\usepackage[breaklinks=true]{hyperref}
\usepackage{tkz-euclide} 
\usepackage{listings}
\usepackage{gvv}       
\usepackage[dvipsnames]{xcolor}
\def\inputGnumericTable{}                                
\usepackage[latin1]{inputenc}                            
\usepackage{color}                                       
\usepackage{array}                                       
\usepackage{longtable}                                   
\usepackage{calc}                                        
\usepackage{multirow}                                    
\usepackage{hhline}                                      
\usepackage{ifthen}                                      
\usepackage{lscape}
\newtheorem{theorem}{Theorem}[section]
\newtheorem{problem}{Problem}
\newtheorem{proposition}{Proposition}[section]
\newtheorem{lemma}{Lemma}[section]
\newtheorem{corollary}[theorem]{Corollary}
\newtheorem{example}{Example}[section]
\newtheorem{definition}[problem]{Definition}
\newcommand{\BEQA}{\begin{eqnarray}}
\newcommand{\EEQA}{\end{eqnarray}}
\newcommand{\define}{\stackrel{\triangle}{=}}
\theoremstyle{remark}
\newtheorem{rem}{Remark}
\begin{document}
\begin{enumerate}
\bibliographystyle{IEEEtran}
\vspace{3cm}
\title{ASSIGNMENT 1}
\author{EE24BTECH11003 - Akshara Sarma Chennubhatla}
\maketitle
\newpage
\bigskip
\section*{\color{black}\colorbox{gray}{\textbf{\large{C:}}}\color{white}\colorbox{magenta}{\textbf{\large{MCQs with One Correct Answer}}}}
\item[20.] If $\alpha + \beta$ = $\frac{\pi}{2}$ and $\beta + \gamma$ = $\alpha$, then $\tan \alpha$ equals
\begin{flushright}
    \brak{\textit{2001S}}
\end{flushright}
(a) 2($\tan \beta$ + $\tan \gamma$)\quad
(b) $\tan\beta$ + $\tan\gamma$\\
(c) $\tan\beta$ + 2 $\tan\gamma$ \quad
(d) 2 $\tan\beta$ + $\tan\gamma$\\\\\\
\item[21.] The number of integral values of k for which the equation $7\cosx + 5\sinx = 2k+1$ has a solution is
\begin{flushright}
    \textcolor{magenta}{\brak{\textit{2002S}}}
\end{flushright}
(a) 4 \quad 
(b) 8 \quad 
(c) 10 \quad 
(d) 12 \quad\\\\\\
\item[22.] Given both $\theta$ and $\phi$ are acute angles and $\sin\theta$ = $\frac{1}{2}$, $\\cos\phi = \frac{1}{3}$, then the value of $\theta + \phi$ belongs to\begin{flushright}
    \textcolor{magenta}{\brak{\textit{2004S}}}
\end{flushright}
(a) $(\frac{\pi}{3},\frac{\pi}{2}]$ \quad
(b) $(\frac{\pi}{2},\frac{2\pi}{3})$ \\\\
(c) $(\frac{2\pi}{3},\frac{5\pi}{6}]$ \quad
(d) $(\frac{5\pi}{6},\pi]$ \\\\\\
\item[\textcolor{magenta}{23.}] $\cos(\alpha - \beta)=1$ and $\cos(\alpha + \beta)=\frac{1}{e}$ where $\alpha, \beta \in [-\pi,\pi]$. Pairs of $\alpha, \beta$ which satisfy both the equations is/are
\begin{flushright}
    \textcolor{magenta}{\brak{\textit{2005S}}}
\end{flushright}
(a) 0 \quad
(b) 1 \quad
(c) 2 \quad
(d) 4 \\\\\\
\item[\textcolor{magenta}{24.}] The values of $\theta \in (0,2\pi)$ for which $2\sin^2\theta - 5\sin\theta + 2 > 0$, are
\begin{flushright}
    \textcolor{magenta}{\brak{\textit{2006-3M,-1}}}
\end{flushright}
(a) $(0,\frac{\pi}{6})\cup(\frac{5\pi}{6},2\pi)$\quad
(b) $(\frac{\pi}{8},\frac{5\pi}{6})$\\\\
(c) $(0,\frac{\pi}{8})\cup(\frac{\pi}{6},\frac{5\pi}{6})$\quad
(d) $(\frac{41\pi}{48},\pi)$\\\\\\
\item[\textcolor{magenta}{25.}] Let $\theta \in (0,\frac{\pi}{4})$ and $t_1 = (\tan\theta)^{\tan\theta}, t_2 = (\tan\theta)^{cot\theta},t_3 = (cot\theta)^{\tan\theta}, t_4 = (cot\theta)^{cot\theta}$, then
\begin{flushright}
    \textcolor{magenta}{\brak{\textit{2006-3M,-1}}}
\end{flushright}
(a) $t_1>t_2>t_3>t_4$\quad
(b) $t_4>t_3>t_1>t_2$\\
(c) $t_3>t_1>t_2>t_4$\quad
(d) $t_2>t_3>t_1>t_4$\\\\\\\\\\
\item[\textcolor{magenta}{26.}] The number of solutions of the pair of equations\\
$2\sin^2\theta - \cos2\theta = 0$\\
$2\cos^2\theta - 3\sin\theta = 0$\\
in the interval $[0,2\pi]$ is
\begin{flushright}
    \textcolor{magenta}{\brak{\textit{2007-3 Marks}}}
\end{flushright}
(a) zero\quad
(b) one\quad
(c) two\quad
(d) four\\\\\\
\item[\textcolor{magenta}{27.}] For $x \in (0,\pi)$, the equation $\sinx + 2\sin2x - \sin3x = 3$ has
\begin{flushright}
    \textcolor{magenta}{\brak{\textit{JEE Adv.2014}}}
\end{flushright}
(a) infinitely many solutions\\
(b) three solutions\\
(c) one solution\\
(d) no solution\\\\\\
\item[\textcolor{magenta}{28.}] Let S = $\{x \in (-\pi,\pi) : x \neq 0, \pm \frac{\pi}{2}\}$. The sum of all distinct solutions of the equation $\sqrt{3}secx + \cosecx + 2(\tanx - cotx) = 0$ in the set S is equal to
\begin{flushright}
    \textcolor{magenta}{\brak{\textit{JEE Adv.2016}}}
\end{flushright}
(a) $-\frac{7\pi}{9}$\quad
(b) $-\frac{2\pi}{9}$\\\\
(c) 0\quad
(d) $\frac{5\pi}{9}$\\\\\\\\\\\\
\item[\textcolor{magenta}{29.}] The value of $\sum_{k=1}^{13}$ $\frac{1}{\sin(\frac{\pi}{4} + \frac{(k-1)\pi}{6})\sin(\frac{\pi}{4} + \frac{k\pi}{6})}$ is equal to
\begin{flushright}
    \textcolor{magenta}{\brak{\textit{JEE Adv.2016}}}
\end{flushright}
(a) $3-\sqrt{3}$\quad
(b) $2(3-\sqrt{3})$\\\\
(c) $2(\sqrt{3}-1)$\quad
(d) $2(2-\sqrt{3})$\\\\\\\\
\section*{\color{black}\colorbox{gray}{\textbf{D:}}\color{white}\colorbox{magenta}{\textbf{MCQs with One or More than One Correct}}}
\item[\textcolor{magenta}{1.}] $(1 + \cos\frac{\pi}{8})(1 + \cos\frac{3\pi}{8})(1 + \cos\frac{5\pi}{8})(1 + \cos\frac{7\pi}{8})$ is equal to
\begin{flushright}
    \textcolor{magenta}{\brak{\textit{1984-3 Marks}}}
\end{flushright}
(a) $\frac{1}{2}$\quad
(b) $\cos \frac{\pi}{8}$\\\\
(c) $\frac{1}{8}$\quad
(d) $\frac{1+\sqrt{2}}{2\sqrt{2}}$\\\\\\
\item[\textcolor{magenta}{2.}] The expression $3[\sin^4(\frac{3\pi}{2} - \alpha) + \sin^4(3\pi + \alpha)] - 2[\sin^6(\frac{\pi}{2} + \alpha) + \sin^6(5\pi - \alpha)]$ is equal to
\begin{flushright}
    \textcolor{magenta}{\brak{\textit{1986-2 Marks}}}
\end{flushright}
(a) 0\quad
(b) 1\\
(c) 3\quad
(d) $\sin4\alpha + \cos6\alpha$\\
(e) none of these\\\\\\
\item[\textcolor{magenta}{3.}] The number of all possible triplets $(a_1, a_2, a_3)$ such that $a_1 + a_2 \cos(2x) + a_3\sin^2(x) = 0$ for all x is
\begin{flushright}
    \textcolor{magenta}{\brak{\textit{1987-2 Marks}}}
\end{flushright}
(a) zero\quad
(b) one\quad
(c) three\\
(d) infinite\quad
(e) none\\\\\\
\item[\textcolor{magenta}{4.}] The values of $\theta$ lying between $\theta = 0$ and $\theta = \frac{\pi}{2}$ and satisfying the\\equation\\\\
\[\begin{vmatrix}
    $$1+\sin^2\theta$$ & $$\cos^2\theta$$ & $$4\sin4\theta$$\\
    $$\sin^2\theta$$ & $$1+\cos^2\theta$$ & $$4\sin4\theta$$\\
    $$\sin^2\theta$$ & $$\cos^2\theta$$ & $$1+4\sin4\theta$$
\end{vmatrix} = 0\] are
\begin{flushright}
    \textcolor{magenta}{\brak{\textit{1988-2 Marks}}}
\end{flushright}
(a) $\frac{7\pi}{24}$\quad
(b) $\frac{5\pi}{24}$\quad
(c) $\frac{11\pi}{24}$\quad
(d) $\frac{\pi}{24}$\\\\\\
\item[\textcolor{magenta}{5.}] Let $2\sin^2x+3\sinx-2>0$ and $x^2-x-2<0$ (x is measured in radians). Then x lies in the interval\begin{flushright}
    \textcolor{magenta}{\brak{\textit{1994}}}
\end{flushright}
(a) $(\frac{\pi}{6},\frac{5\pi}{6})$\quad
(b) $(-1,\frac{5\pi}{6})$\\\\
(c) $(-1,2)$\quad
(d) $(\frac{\pi}{6},2)$
\end{enumerate}
\end{document}

